%% Generated by Sphinx.
\def\sphinxdocclass{jupyterBook}
\documentclass[letterpaper,10pt,english]{jupyterBook}
\ifdefined\pdfpxdimen
   \let\sphinxpxdimen\pdfpxdimen\else\newdimen\sphinxpxdimen
\fi \sphinxpxdimen=.75bp\relax
\ifdefined\pdfimageresolution
    \pdfimageresolution= \numexpr \dimexpr1in\relax/\sphinxpxdimen\relax
\fi
%% let collapsible pdf bookmarks panel have high depth per default
\PassOptionsToPackage{bookmarksdepth=5}{hyperref}
%% turn off hyperref patch of \index as sphinx.xdy xindy module takes care of
%% suitable \hyperpage mark-up, working around hyperref-xindy incompatibility
\PassOptionsToPackage{hyperindex=false}{hyperref}
%% memoir class requires extra handling
\makeatletter\@ifclassloaded{memoir}
{\ifdefined\memhyperindexfalse\memhyperindexfalse\fi}{}\makeatother

\PassOptionsToPackage{warn}{textcomp}

\catcode`^^^^00a0\active\protected\def^^^^00a0{\leavevmode\nobreak\ }
\usepackage{cmap}
\usepackage{fontspec}
\defaultfontfeatures[\rmfamily,\sffamily,\ttfamily]{}
\usepackage{amsmath,amssymb,amstext}
\usepackage{polyglossia}
\setmainlanguage{english}



\setmainfont{FreeSerif}[
  Extension      = .otf,
  UprightFont    = *,
  ItalicFont     = *Italic,
  BoldFont       = *Bold,
  BoldItalicFont = *BoldItalic
]
\setsansfont{FreeSans}[
  Extension      = .otf,
  UprightFont    = *,
  ItalicFont     = *Oblique,
  BoldFont       = *Bold,
  BoldItalicFont = *BoldOblique,
]
\setmonofont{FreeMono}[
  Extension      = .otf,
  UprightFont    = *,
  ItalicFont     = *Oblique,
  BoldFont       = *Bold,
  BoldItalicFont = *BoldOblique,
]



\usepackage[Bjarne]{fncychap}
\usepackage[,numfigreset=1,mathnumfig]{sphinx}

\fvset{fontsize=\small}
\usepackage{geometry}


% Include hyperref last.
\usepackage{hyperref}
% Fix anchor placement for figures with captions.
\usepackage{hypcap}% it must be loaded after hyperref.
% Set up styles of URL: it should be placed after hyperref.
\urlstyle{same}


\usepackage{sphinxmessages}



        % Start of preamble defined in sphinx-jupyterbook-latex %
         \usepackage[Latin,Greek]{ucharclasses}
        \usepackage{unicode-math}
        % fixing title of the toc
        \addto\captionsenglish{\renewcommand{\contentsname}{Contents}}
        \hypersetup{
            pdfencoding=auto,
            psdextra
        }
        % End of preamble defined in sphinx-jupyterbook-latex %
        

\title{Mathematical Modeling in Physics}
\date{Sep 06, 2022}
\release{}
\author{Danny Caballero}
\newcommand{\sphinxlogo}{\vbox{}}
\renewcommand{\releasename}{}
\makeindex
\begin{document}

\pagestyle{empty}
\sphinxmaketitle
\pagestyle{plain}
\sphinxtableofcontents
\pagestyle{normal}
\phantomsection\label{\detokenize{content/intro::doc}}


\sphinxAtStartPar
PHY 415, called, “Mathematical Methods for Physicists” is a course the brings together many of the mathematical approaches that we commonly use in physics and apply them to variety of problems. In this course, we will take a modeling\sphinxhyphen{}based approach where we focus on the mathematical descriptions of physical phenomenon and determine what mathematical and analytical approaches are useful in exploring those models.

\begin{sphinxadmonition}{note}{Updates}

\sphinxAtStartPar
\sphinxstylestrong{05 Sept 22}
\begin{itemize}
\item {} 
\sphinxAtStartPar
Tuesday’s \sphinxhref{https://dannycaballero.info/phy415msu/content/2\_oscillations/activity-SHO\_and\_dynamical\_systems.html}{in\sphinxhyphen{}class activity} has been posted.
\begin{itemize}
\item {} 
\sphinxAtStartPar
Review as you see fit; we will go over the whole thing together

\end{itemize}

\end{itemize}

\sphinxAtStartPar
\sphinxstylestrong{02 Sept 22}
\begin{itemize}
\item {} 
\sphinxAtStartPar
\sphinxhref{https://www.when2meet.com/?16627934-Bt7PQ}{Complete the survey} by Tuesday morning to help us schedule the Project Help Sessions

\item {} 
\sphinxAtStartPar
Synthesized notes from 01 Sept 22 class are {\hyperref[\detokenize{content/1_modeling/notes-what_is_a_model::doc}]{\sphinxcrossref{\DUrole{doc,std,std-doc}{posted}}}}

\item {} 
\sphinxAtStartPar
Readings have been updated for next week: {\hyperref[\detokenize{content/2_oscillations/readings-oscillators::doc}]{\sphinxcrossref{\DUrole{doc,std,std-doc}{Read Oscillations}}}}

\item {} 
\sphinxAtStartPar
Make sure to complete the {\hyperref[\detokenize{content/0_course/reading_questions::doc}]{\sphinxcrossref{\DUrole{doc,std,std-doc}{Reading Questions Form}}}} before class Tuesday morning

\item {} 
\sphinxAtStartPar
More project details are coming, a draft project rubric is forthcoming, and activities for class this coming week will be posted by Monday

\end{itemize}
\end{sphinxadmonition}

\sphinxAtStartPar
To get a sense of the course, please read all the pages associated with our syllabus.

\begin{DUlineblock}{0em}
\item[] \sphinxstylestrong{\Large Learning Objectives}
\end{DUlineblock}

\sphinxAtStartPar
In this course, you will learn to:
\begin{itemize}
\item {} 
\sphinxAtStartPar
investigate physical systems using a variety of tools and approaches,

\item {} 
\sphinxAtStartPar
construct and document a reproducible process for those investigations,

\item {} 
\sphinxAtStartPar
use analytical, computational, and graphical approaches to answer specific questions in those investigations,

\item {} 
\sphinxAtStartPar
provide evidence of the quality of work using a variety of sources, and

\item {} 
\sphinxAtStartPar
collaborate effectively and contribute to a inclusive learning environment

\end{itemize}

\sphinxstepscope


\chapter{0 \sphinxhyphen{} Overview of PHY 415}
\label{\detokenize{content/0_course/syllabus:overview-of-phy-415}}\label{\detokenize{content/0_course/syllabus::doc}}
\sphinxAtStartPar
In designing this course, I plan to emphasize more independent learning on your part and greater agency for you in determining what you learn and how you demonstrate you have learned. So you should expect:
\begin{itemize}
\item {} 
\sphinxAtStartPar
to read a variety of pieces of information to coordinate information

\item {} 
\sphinxAtStartPar
to present your ideas publicly and to discuss them

\item {} 
\sphinxAtStartPar
to learn new approaches and novel techniques on your own

\item {} 
\sphinxAtStartPar
to become more expert than me in the areas of your interest

\item {} 
\sphinxAtStartPar
to learn more about scientists that you have not learned about

\end{itemize}

\sphinxAtStartPar
This is not to say that you are on your own. Here’s what you can expect from me:
\begin{itemize}
\item {} 
\sphinxAtStartPar
resources, information, and tools to help you learn

\item {} 
\sphinxAtStartPar
support and scaffolding to move you towards more independence in your learning

\item {} 
\sphinxAtStartPar
timely and detailed feedback to help you along

\item {} 
\sphinxAtStartPar
a commitment to an inclusive classroom

\end{itemize}


\section{Contact Information}
\label{\detokenize{content/0_course/syllabus:contact-information}}

\subsection{Instructor}
\label{\detokenize{content/0_course/syllabus:instructor}}\begin{itemize}
\item {} 
\sphinxAtStartPar
\sphinxhref{http://dannycab.github.io}{Prof. Danny Caballero} (he/him/his)

\item {} 
\sphinxAtStartPar
Class Meetings: Tuesdays and Thursdays 10:20am\sphinxhyphen{}12:10pm (Location: 1300 BPS)

\item {} 
\sphinxAtStartPar
Email: \sphinxhref{mailto:caball14@msu.edu}{caball14@msu.edu}, office: 1310\sphinxhyphen{}A BPS

\item {} 
\sphinxAtStartPar
Office hrs: To be scheduled, but I also have an open door policy. I enjoy visiting and talking with you about physics.

\item {} 
\sphinxAtStartPar
Web page:
\sphinxhref{http://dannycab.github.io/phy415msu/}{dannycab.github.io/phy415msu/}

\end{itemize}


\subsection{Teaching Assistant}
\label{\detokenize{content/0_course/syllabus:teaching-assistant}}\begin{itemize}
\item {} 
\sphinxAtStartPar
Neshad Amarasinghe (he/him/his)

\item {} 
\sphinxAtStartPar
Email: \sphinxhref{mailto:devanes1@msu.edu}{devanes1@msu.edu}

\end{itemize}


\subsection{Slack Team}
\label{\detokenize{content/0_course/syllabus:slack-team}}
\sphinxAtStartPar
\sphinxstyleemphasis{This term we will be using Slack for class discussion.} The system is highly catered to getting you help fast and efficiently from classmates and myself. Rather than emailing questions, I encourage you to post your questions on there. You should be added to the Slack team by the first week of class. Email Danny if you are not part of the team.

\sphinxAtStartPar
\sphinxstylestrong{Link:} \sphinxurl{https://phy415fall2022.slack.com/}


\section{Grading}
\label{\detokenize{content/0_course/syllabus:grading}}
\sphinxAtStartPar
Details about {\hyperref[\detokenize{content/0_course/design::doc}]{\sphinxcrossref{\DUrole{doc,std,std-doc}{course activities are here}}}} and {\hyperref[\detokenize{content/0_course/assessments::doc}]{\sphinxcrossref{\DUrole{doc,std,std-doc}{information regarding assessment is here}}}}. Your grade will be comprised of weekly discussion questions and several projects that you will complete in the form of a Jupyter notebook (a “computational essay”, which we will discuss later). Your grade on each project is split between completion (50\%) and quality (50\%). We will collectively define “quality” in class, but we have provided {\hyperref[\detokenize{content/0_course/rubric::doc}]{\sphinxcrossref{\DUrole{doc,std,std-doc}{a preliminary rubric}}}} for us to work from for the first project. Your final grade will be scaled based on your best performances; there will be slightly more projects than what comprises your grade.  \sphinxstyleemphasis{The intent here is to to allow you space to explore a model or project that you really enjoy, and to reward you for doing that.} How your grade is calculated appears below.


\begin{savenotes}\sphinxattablestart
\centering
\begin{tabulary}{\linewidth}[t]{|T|T|}
\hline
\sphinxstyletheadfamily 
\sphinxAtStartPar
Activity
&\sphinxstyletheadfamily 
\sphinxAtStartPar
Percent of Grade
\\
\hline
\sphinxAtStartPar
Best Project Grade
&
\sphinxAtStartPar
30\%
\\
\hline
\sphinxAtStartPar
2nd Best Project Grade
&
\sphinxAtStartPar
25\%
\\
\hline
\sphinxAtStartPar
3rd Best Project Grade
&
\sphinxAtStartPar
20\%
\\
\hline
\sphinxAtStartPar
4th Best Project Grade
&
\sphinxAtStartPar
10\%
\\
\hline
\sphinxAtStartPar
Weekly Discussion Questions (completion)
&
\sphinxAtStartPar
15\%
\\
\hline
\end{tabulary}
\par
\sphinxattableend\end{savenotes}

\sphinxAtStartPar
\sphinxstylestrong{While attendance is not required, you are unlikely to succeed with your projects without regular attendance and engagement.}

\sphinxstepscope


\section{Course Objectives}
\label{\detokenize{content/0_course/goals:course-objectives}}\label{\detokenize{content/0_course/goals::doc}}
\sphinxAtStartPar
This course emphasizes making models of physical phenomenon and how we use various tools at our disposal to investigate those models. Hence, we have learning objectives for making models of these systems and for learning specific tools.

\begin{sphinxadmonition}{note}{Principle Learning Objectives}

\sphinxAtStartPar
Students will demonstrate they can:
\begin{itemize}
\item {} 
\sphinxAtStartPar
investigate physical systems of their choosing using a variety of tools and approaches

\item {} 
\sphinxAtStartPar
construct and document a reproducible process for those investigations

\item {} 
\sphinxAtStartPar
use analytical, computational, and graphical approaches to answer specific questions in those investigations

\item {} 
\sphinxAtStartPar
provide evidence of the quality of their work using a variety of sources

\item {} 
\sphinxAtStartPar
collaborate effectively and contribute to a inclusive learning environment

\end{itemize}
\end{sphinxadmonition}

\sphinxAtStartPar
Each of these learning objectives contributes to your development as a physicist. I recognize that these are \sphinxstylestrong{big} ideas to think about. What I mean is that the objectives above are quite broad and you might be able to see a little about what or why they are included. But, below, I added more detail about each one along with a smaller scale list of objectives that you will engage with. Throughout our course, you will have opportunities to demonstrate these objectives in your work. \sphinxstyleemphasis{My aim is to make what you are assessed on in this course something you are interested in, so these objectives reflect that.}


\subsection{Investigate physical systems}
\label{\detokenize{content/0_course/goals:investigate-physical-systems}}
\sphinxAtStartPar
Clearly, one of our central goals is learning how to make models of physical systems. This means learning about and developing fluency with a wide variety of mathematical and computational tools. In this courses, we will make extensive use of \sphinxhref{http://anaconda.org}{Jupyter notebooks} for homework and projects. In fact, what you are reading is a set of Jupyter notebooks! Below, you will see the list of objectives for this principal objective.

\begin{sphinxadmonition}{note}{Investigating Physical Systems Learning Objectives}

\sphinxAtStartPar
Students will demonstrate they can:
\begin{itemize}
\item {} 
\sphinxAtStartPar
use mathematical techniques to predict or explain some physical phenomenon

\item {} 
\sphinxAtStartPar
employ computational models and algorithms to investigate physical systems

\item {} 
\sphinxAtStartPar
compare analytical and computational approaches to these investigations

\item {} 
\sphinxAtStartPar
provide coherent explanations for their investigations buttressed by physical, mathematical, and/or computational knowledge and principles

\end{itemize}
\end{sphinxadmonition}


\subsection{Construct and document a reproducible process}
\label{\detokenize{content/0_course/goals:construct-and-document-a-reproducible-process}}
\sphinxAtStartPar
A critical element of physics work is making sure that with the same setup and approach, others can reproduce the work you have done. This provides validity to your work and evidences how we develop collective understanding of physics. Physics is a social enterprise and the ensuring the reproducibility of work supports that enterprise. Below are the learning objectives for this principal objective.

\begin{sphinxadmonition}{note}{Reproducibility Learning Objectives}

\sphinxAtStartPar
Students will demonstrate they can:
\begin{itemize}
\item {} 
\sphinxAtStartPar
document their work and analysis such that others can reproduce their work

\item {} 
\sphinxAtStartPar
consistently reproduce their work and results in a variety of contexts

\item {} 
\sphinxAtStartPar
provide an explanation for why certain work or results are not (or should not be) reproducible

\end{itemize}
\end{sphinxadmonition}


\subsection{Use analytical, computational, and graphical approaches}
\label{\detokenize{content/0_course/goals:use-analytical-computational-and-graphical-approaches}}
\sphinxAtStartPar
The main approaches that we use to make models are mathematical, computational, and graphical. In this class, we will aim to leverage the benefits of each to learn more about the physical systems that we are investigating. Indeed, much of the “knowledge” that you are going to develop will be about specific analytical, computational, or graphical approaches to investigate physical systems. Below are the learning objectives for this principal objective.

\begin{sphinxadmonition}{note}{Modeling Approaches Learning Objectives}

\sphinxAtStartPar
Students will demonstrate they can:
\begin{itemize}
\item {} 
\sphinxAtStartPar
Use a wide variety of modeling techniques to investigate different physical systems

\item {} 
\sphinxAtStartPar
Choose and employ appropriate approaches to modeling physical systems of their choosing

\item {} 
\sphinxAtStartPar
Explain how those approaches lead to different results or conclusions

\end{itemize}
\end{sphinxadmonition}


\subsection{Provide evidence of the quality of their work}
\label{\detokenize{content/0_course/goals:provide-evidence-of-the-quality-of-their-work}}
\sphinxAtStartPar
The definition of the quality of a piece of science is a collective decision by the scientific community. In established communities, like physics, there are commonly\sphinxhyphen{}accepted ways of defining the quality of work (norms, customs, and rules all play a role). But that is not to mean those ways can’t change; papers describing quantum physics and relativity brushed up hard against this issue of quality and were both dismissed and celebrated. Newer disciplines are still establishing those norms and rules. And in some cases, disciplines are pushing back against Western norms of quality. In our class, we will collectively decide what we mean by ‘’high quality’’ work. Below are the learning objectives for this principal objective.

\begin{sphinxadmonition}{note}{Quality Control Learning Objectives}

\sphinxAtStartPar
Students will demonstrate they can:
\begin{itemize}
\item {} 
\sphinxAtStartPar
describe what it means to have high quality work in our class

\item {} 
\sphinxAtStartPar
look for and evaluate when work meets those standards

\item {} 
\sphinxAtStartPar
provide suggestions (or act on suggestions) to improve the quality of their work

\end{itemize}
\end{sphinxadmonition}


\subsection{Collaborate effectively}
\label{\detokenize{content/0_course/goals:collaborate-effectively}}
\sphinxAtStartPar
Physics is a social enterprise that relies on effective and productive collaborations. Very little (if any) science is done alone; the scale of science is too grand for individuals to effectively work – everyone needs a team. In this spirit, in this classroom, we deeply encourage collaboration. We will try to develop effective collaboration through your work on projects and our in\sphinxhyphen{}class activities. Below are the learning objectives for this principal objective.

\begin{sphinxadmonition}{note}{Collaboration Learning Objectives}

\sphinxAtStartPar
Students will demonstrate they can:
\begin{itemize}
\item {} 
\sphinxAtStartPar
Collaborate on a variety of activities in and out of class

\item {} 
\sphinxAtStartPar
Document the contributions in these collaborations and make changes if contributions are unbalanced

\item {} 
\sphinxAtStartPar
Develop personally effective strategies for collaboration

\end{itemize}
\end{sphinxadmonition}

\sphinxstepscope


\section{Course Design}
\label{\detokenize{content/0_course/design:course-design}}\label{\detokenize{content/0_course/design::doc}}
\sphinxAtStartPar
For most of you, 4415 is an elective course that you are taking to learn more about how we use mathematical techniques in physics. As such, this course is designed under several different principles than a standard course. Below, I provide those principles and their rationale.
\begin{itemize}
\item {} 
\sphinxAtStartPar
415 should help you learn the central tenets of modeling physical systems
\begin{itemize}
\item {} 
\sphinxAtStartPar
The sheer volume of mathematical and computational physics knowledge out there is immense and impossible for any one person to learn. However, the central elements of making models, how to learn about specific techniques, and how to debug your approaches are things we can learn and employ broadly as well as to specific problems.

\end{itemize}

\item {} 
\sphinxAtStartPar
415 should be a celebration of your knowledge
\begin{itemize}
\item {} 
\sphinxAtStartPar
For most of you, this course is part of your senior level coursework. What you have achieved in the last three to four years should be celebrated and enjoyed. This course will provide ample opportunities for you to share what things you know and what things you are learning with me and with each other.

\end{itemize}

\item {} 
\sphinxAtStartPar
415 should give you opportunities to engage in professional practice
\begin{itemize}
\item {} 
\sphinxAtStartPar
As you start towards your professional career, it’s important to learn what professional scientists do. You have probably already begun this work in advanced lab and research projects that you have worked on. We will continue developing your professional skills in this course through the use of course projects.

\end{itemize}

\item {} 
\sphinxAtStartPar
415 will illustrate that we can learn from each other
\begin{itemize}
\item {} 
\sphinxAtStartPar
Even though I’ve been learning physics for almost 20 years, I don’t know everything. I am excited to learn from you and I hope that you are excited to learn from me and each other.

\end{itemize}

\end{itemize}


\subsection{Optional purchases:}
\label{\detokenize{content/0_course/design:optional-purchases}}
\sphinxAtStartPar
The core readings and work for this course will be this jupyterbook. I will find resources online, make my own, and provide as much organized free material as possible. If you want to have a textbook that helps you organize your readings, please obtain copies of:
\begin{enumerate}
\sphinxsetlistlabels{\arabic}{enumi}{enumii}{}{.}%
\item {} 
\sphinxAtStartPar
Mary Boas, \sphinxhref{https://www.amazon.com/Mathematical-Methods-Physical-Sciences-Mary/dp/04711982}{\sphinxstyleemphasis{Mathematical Methods in the Physical Sciences}} (Wiley; 2005). This book is the definitive text on mathematical approaches, written by Dr. Boas originally in 1966! Any 3rd edition will be useful and I will put the section numbers from Boas in the online readings.

\item {} 
\sphinxAtStartPar
Mark Newman, \sphinxhref{https://www.amazon.com/Computational-Physics-Mark-Newman/dp/1480145513}{\sphinxstyleemphasis{Computational Physics}} (CreateSpace Independent Publishing Platform; 2012). This book is a great introduction to a variety of computational physics techniques, written by UMich professor Mark Newman for a computational physics course. I will put section numbers from Newman in the online readings.

\end{enumerate}


\subsubsection{Additional sources}
\label{\detokenize{content/0_course/design:additional-sources}}
\sphinxAtStartPar
In addition, I will draw from the following books. I have copies of them if you want or need scans of sections. But they can found online in Google Books and other places as well. no need to purchase unless you want a copy for your personal library.


\paragraph{Mechanics}
\label{\detokenize{content/0_course/design:mechanics}}\begin{itemize}
\item {} 
\sphinxAtStartPar
Edwin Taylor, Mechanics

\item {} 
\sphinxAtStartPar
Jerry Marion and Stephen Thornton, Classical Dynamics of Particles and Systems

\item {} 
\sphinxAtStartPar
Charles Kittel, Walter D. Knight, Malvin A. Ruderman, A. Carl Helholtz, and Burton J. Moyer, Mechanics

\end{itemize}


\paragraph{Electromagnetism}
\label{\detokenize{content/0_course/design:electromagnetism}}\begin{itemize}
\item {} 
\sphinxAtStartPar
Edward Purcell, Electricity and Magnetism

\item {} 
\sphinxAtStartPar
David J. Grriffths, Introduction to Electromagnetism

\end{itemize}


\paragraph{Quantum Mechanics}
\label{\detokenize{content/0_course/design:quantum-mechanics}}\begin{itemize}
\item {} 
\sphinxAtStartPar
David McIntyre, Quantum Mechanics

\item {} 
\sphinxAtStartPar
David J. Griffiths, Introduction to Quantum Mechanics

\end{itemize}


\paragraph{Waves and Thermal Physics}
\label{\detokenize{content/0_course/design:waves-and-thermal-physics}}\begin{itemize}
\item {} 
\sphinxAtStartPar
Frank S. Crawford, Waves

\item {} 
\sphinxAtStartPar
Charles Kittel, Thermal Physics

\item {} 
\sphinxAtStartPar
Ashley Carter, Classical and Statistical Thermodynamics

\item {} 
\sphinxAtStartPar
Daniel Schroeder, Thermal Physics

\end{itemize}


\paragraph{Additional Physics Topics}
\label{\detokenize{content/0_course/design:additional-physics-topics}}\begin{itemize}
\item {} 
\sphinxAtStartPar
Steven H. Strogatz, Nonlinear Dynamics and Chaos

\item {} 
\sphinxAtStartPar
B Lautrup, Physics of Continuous Matter

\item {} 
\sphinxAtStartPar
Frank L. Pedrotti and Leno S. Pedrotti, Introduction to Optics

\end{itemize}


\paragraph{Mathematics}
\label{\detokenize{content/0_course/design:mathematics}}\begin{itemize}
\item {} 
\sphinxAtStartPar
Susan M. Lea, Mathematics for Physicists

\item {} 
\sphinxAtStartPar
William E. Boyce and Richard C. DiPrima, Elementary Differential Equations

\item {} 
\sphinxAtStartPar
James Brown and Ruel Churchill, Complex Variables and Applications

\item {} 
\sphinxAtStartPar
Jerrold Marsden and Anthony Tromba, Vector Calculus

\item {} 
\sphinxAtStartPar
Sheldon Ross, A First Course in Probability

\end{itemize}


\paragraph{Presenting (Visual) Information}
\label{\detokenize{content/0_course/design:presenting-visual-information}}\begin{itemize}
\item {} 
\sphinxAtStartPar
Edward Tufte, The Visual Display of Quantitative information

\item {} 
\sphinxAtStartPar
Albert Cairo, The Truthful Art

\item {} 
\sphinxAtStartPar
Stephen E. Toulmin, The Uses of Argument

\end{itemize}


\section{Course Activities}
\label{\detokenize{content/0_course/design:course-activities}}

\subsection{“Readings”}
\label{\detokenize{content/0_course/design:readings}}
\sphinxAtStartPar
\sphinxstylestrong{“Reading”} is an essential part of 415! Reading the notes before class is very important. I use “reading” in quotes, because in our class this idea goes beyond just reading text and includes understanding figures and watching videos. These should help inform the basis of your understating that we will draw on in class to clarify your understanding and to help you make sense of the material. I will assume you have done the required readings in advance! It will make a huge difference if you spend the time and effort to carefully read and follow the resources posted. The {\hyperref[\detokenize{content/0_course/calendar::doc}]{\sphinxcrossref{\DUrole{doc,std,std-doc}{calendar}}}} has the details on videos and readings that will be updated.

\sphinxAtStartPar
\sphinxstylestrong{Weekly Questions}: To encourage and reward you for keeping up with the “readings”, there will be weekly questions about the readings posted for you to respond to. These are not meant to test your knowledge, but rather to focus your “reading” towards what you understand, and what you don’t yet understand. I will ask you about those things weekly and use that information to tailor in\sphinxhyphen{}class activities based on what I am hearing is confusing, unclear, or challenging. These questions are only graded for completion, but I do want your honest attempt. You can {\hyperref[\detokenize{content/0_course/reading_questions::doc}]{\sphinxcrossref{\DUrole{doc,std,std-doc}{review the questions now}}}}.


\subsection{Class Meetings}
\label{\detokenize{content/0_course/design:class-meetings}}
\sphinxAtStartPar
\sphinxstylestrong{Classroom Etiquette:} Please silence your electronic devices when entering the classroom. I don’t mind you using them (in fact, see below, we will use them). But, sometimes, they can very distracting to your neighbors, so use your judgement. I appreciate that you might have questions or comments about things in class. We are going to be having short lectures combined with longer project work in class. So you will have plenty of time to catch up with social media and the news.

\sphinxAtStartPar
If you and/or your group mates are confused, just raise your hand and ask questions. If you are confused, you are likely not the only one and it’s better to chat about it, then move on. Questions are always good, and are strongly encouraged! \sphinxstyleemphasis{The only way we learn is to question what we know and how we know it.}

\sphinxAtStartPar
\sphinxstylestrong{Computing Devices:} Please bring some sort of computing device to class everyday. You might be researching information online, reviewing work you have done, or actively building models of systems together. This device can be a computer, a tablet, or a phone. You can also partner up with folks because we will use them in groups. \sphinxstyleemphasis{If you need a computing device brought to class for you or your group mates to use, let me know. I will organize for some small collection of laptops if we need it.}

\sphinxAtStartPar
\sphinxstylestrong{In\sphinxhyphen{}Class:} We will have some short lectures about topics or concepts; some of those will be in\sphinxhyphen{}the\sphinxhyphen{}moment as needed. The idea is that you are developing a basic understanding through readings and videos, practicing using those new ideas with me and with your classmates in class, and then applying what you are learning to new ideas. So, we will also use a variety of in\sphinxhyphen{}class activities that help you construct an understanding of a particular topic or concept. These will not be collected or graded, but we will discuss the solutions in class. \sphinxstyleemphasis{I will not post solutions for these activties as we have no exams or quizzes.}


\subsection{Projects}
\label{\detokenize{content/0_course/design:projects}}
\sphinxAtStartPar
\sphinxstylestrong{In\sphinxhyphen{}class Projects:} The class is designed to support your independent research into ideas that you are excited about. So in\sphinxhyphen{}class projects are meant to equip you with the knowledge and practice to learn new things for your projects. These in\sphinxhyphen{}class projects will be short demonstrations of models that you complete in groups. We will circulate around the room and check on you and your group’s progress and understanding. At the end of the class period, we will share the results of the in\sphinxhyphen{}class project and discuss any sticking points. These in\sphinxhyphen{}class activites will not be graded, but they will be essential for your out\sphinxhyphen{}of\sphinxhyphen{}class projects.

\sphinxAtStartPar
\sphinxstylestrong{Out\sphinxhyphen{}of\sphinxhyphen{}class Projects:} For this class, we anticipate 6 projects to be turned in roughly every 2\sphinxhyphen{}3 weeks, with a weeklong turn\sphinxhyphen{}in window (see {\hyperref[\detokenize{content/0_course/calendar::doc}]{\sphinxcrossref{\DUrole{doc,std,std-doc}{calendar}}}}). Except for the first project, up to 3 of these projects can be completed as partner projects. Partner projects are subject to a different grading rubric that evaluates collaborative efforts and increases the expectation for other areas compared to an individual project. A preliminary rubric appears {\hyperref[\detokenize{content/0_course/rubric::doc}]{\sphinxcrossref{\DUrole{doc,std,std-doc}{here}}}}, but we will define these collectively after the first project.

\sphinxAtStartPar
These projects will take the form a \sphinxhref{https://uio-ccse.github.io/computational-essay-showroom/}{computational essay}, which provides documentation and rationale for the exploration that you are completing. We will model a computational essay project in our first project and we will reflect on the {\hyperref[\detokenize{content/0_course/rubric::doc}]{\sphinxcrossref{\DUrole{doc,std,std-doc}{rubric}}}} after it, and make changes collectively as a class to it.

\sphinxAtStartPar
\sphinxstyleemphasis{I strongly encourage collaboration}, an essential skill in science and engineering (and highly valued by employers!) Social interactions are critical to scientists’ success – most good ideas grow out of discussions with colleagues, and essentially all physicists work as part of a group. Find partners and work together. However, it is also important that you OWN the material. I strongly suggest you start working by yourself (and that means really making an extended effort on every activity). Then work with a group, and finally, finish up on your own – write up your own work, in your own way. There will also be time for peer discussion during classes – as you work together, try to help your partners get over confusions, listen to them, ask each other questions, critique, teach each other. You will learn a lot this way! For all assignments, the work you turn in must in the end be your own: in your own words, reflecting your own understanding. (If, at any time, for any reason, you feel disadvantaged or isolated, contact me and I can discretely try to help arrange study groups.)


\subsubsection{Help Session}
\label{\detokenize{content/0_course/design:help-session}}
\sphinxAtStartPar
Help sessions/office hours are to facilitate your learning. We encourage attendance \sphinxhyphen{} plan on working in small groups, our role will be as learning coaches. The sessions are concept and project\sphinxhyphen{}centric, but we will not be explicitly telling anyone how to do your project (how would that help you learn?) I strongly encourage you to start all projects on your own. If you come to help sessions “cold”, the value of the project to you will be greatly reduced.

\sphinxAtStartPar
\sphinxstylestrong{Please report your preferred times for help session:}

\sphinxAtStartPar
\sphinxstylestrong{QR Code for form}

\sphinxAtStartPar
\sphinxincludegraphics{{help_session_qr}.png}

\sphinxAtStartPar
\sphinxstylestrong{Link to form}

\sphinxAtStartPar
\sphinxurl{https://www.when2meet.com/?16627934-Bt7PQ}

\sphinxstepscope


\section{Assessments}
\label{\detokenize{content/0_course/assessments:assessments}}\label{\detokenize{content/0_course/assessments::doc}}

\subsection{Formative Assessment}
\label{\detokenize{content/0_course/assessments:formative-assessment}}
\sphinxAtStartPar
Formative assessment is often ungraded and reflective assessment. It is meant to help you make changes to your thinking, approaches, or practice. It is not evaluative, it’s corrective; to help you make changes. We will make heavy use of ungraded formative feedback throughout the course.


\subsection{Summative Assessment}
\label{\detokenize{content/0_course/assessments:summative-assessment}}
\sphinxAtStartPar
Summative assessment is typically evaluative and will take the form of course projects completed out of class. These projects will take the form of a computational essay in which you write mathematics and code to investigate and explain a given phenomenon of interest. We will explore those essays in class and talk about what makes a useful one as we define a rubric for evaluation.


\subsubsection{Preliminary Rubric}
\label{\detokenize{content/0_course/assessments:preliminary-rubric}}
\sphinxAtStartPar
{\hyperref[\detokenize{content/0_course/rubric::doc}]{\sphinxcrossref{\DUrole{doc,std,std-doc}{A preliminary rubric}}}} has been posted. We will use this rubric for the first out\sphinxhyphen{}of\sphinxhyphen{}class project evaluation. We will then reflect on it and make changes to collectively as a class.


\subsubsection{Resources for Computational Essays}
\label{\detokenize{content/0_course/assessments:resources-for-computational-essays}}
\sphinxAtStartPar
If you want to read more about computational essays, here’s a few links in the order utility/readability:
\begin{itemize}
\item {} 
\sphinxAtStartPar
Steven Wolfram \sphinxhyphen{} \sphinxhref{https://writings.stephenwolfram.com/2017/11/what-is-a-computational-essay/}{What is a Computational Essay?}

\item {} 
\sphinxAtStartPar
University of Oslo Physics \sphinxhyphen{} \sphinxhref{https://uio-ccse.github.io/computational-essay-showroom/}{Examples of Computational Essays}

\item {} 
\sphinxAtStartPar
Odden and Burk, The Physics Teacher \sphinxhyphen{} \sphinxhref{https://aapt.scitation.org/doi/abs/10.1119/1.5145471}{Computational Essays in the Physics Classroom}

\item {} 
\sphinxAtStartPar
Odden, Lockwood, and Caballero, Physical Review PER \sphinxhyphen{} \sphinxhref{https://journals.aps.org/prper/abstract/10.1103/PhysRevPhysEducRes.15.020152}{Physics computational literacy: An exploratory case study using computational essays}

\end{itemize}

\sphinxstepscope


\section{Project Rubrics}
\label{\detokenize{content/0_course/rubric:project-rubrics}}\label{\detokenize{content/0_course/rubric::doc}}

\subsection{Preliminary (For first out of class project)}
\label{\detokenize{content/0_course/rubric:preliminary-for-first-out-of-class-project}}
\sphinxAtStartPar
Our rubric will initially focus on the {\hyperref[\detokenize{content/0_course/goals::doc}]{\sphinxcrossref{\DUrole{doc,std,std-doc}{core learning goals}}}}. As we develop expertise in these areas, we might decide to change emphases or focus.


\begin{savenotes}\sphinxattablestart
\centering
\begin{tabulary}{\linewidth}[t]{|T|T|}
\hline
\sphinxstyletheadfamily 
\sphinxAtStartPar
Goal
&\sphinxstyletheadfamily 
\sphinxAtStartPar
Fractional Importance
\\
\hline
\sphinxAtStartPar
Investigate physical systems
&
\sphinxAtStartPar
0.30
\\
\hline
\sphinxAtStartPar
Construct and document a reproducible process
&
\sphinxAtStartPar
0.10
\\
\hline
\sphinxAtStartPar
Use analytical, computational, and graphical approaches
&
\sphinxAtStartPar
0.30
\\
\hline
\sphinxAtStartPar
Provide evidence of the quality of their work
&
\sphinxAtStartPar
0.10
\\
\hline
\sphinxAtStartPar
Collaborate effectively
&
\sphinxAtStartPar
0.20
\\
\hline
\end{tabulary}
\par
\sphinxattableend\end{savenotes}

\sphinxAtStartPar
I will develop specific rubric elements based on your engagement with the in\sphinxhyphen{}class materials on the first couple of class periods. I need to observe where we need the most support initially and that will change the rubric expectations.

\sphinxstepscope


\section{Calendar}
\label{\detokenize{content/0_course/calendar:calendar}}\label{\detokenize{content/0_course/calendar::doc}}
\sphinxAtStartPar
In this course, we will cover three principal topics in physics (in this order):
\begin{itemize}
\item {} 
\sphinxAtStartPar
{\hyperref[\detokenize{content/2_oscillations/readings-oscillators::doc}]{\sphinxcrossref{\DUrole{doc,std,std-doc}{Oscillations}}}}

\item {} 
\sphinxAtStartPar
Fields

\item {} 
\sphinxAtStartPar
Distributions

\end{itemize}

\sphinxAtStartPar
We will spend roughly 1/3 of the course on each, and the expectaitn is that each topic has 2 completed projects.

\sphinxAtStartPar
A Google Calendar appears below for class. If you review the notes in a given event, you will find the details for a class or the readings to do.



\sphinxstepscope


\section{Classroom Environment}
\label{\detokenize{content/0_course/environment:classroom-environment}}\label{\detokenize{content/0_course/environment::doc}}

\subsection{Commitment to an Inclusive Classroom}
\label{\detokenize{content/0_course/environment:commitment-to-an-inclusive-classroom}}
\sphinxAtStartPar
I am deeply committed to creating an inclusive classroom \sphinxhyphen{} one where you and your classmates
feel comfortable, intellectually challenged, and able to speak up about your ideas
and experiences. This means that our classroom, our virtual environments, and our interactions
need to be as inclusive as possible. Mutual respect, civility, and the ability to listen
and observe others are central to creating a classroom that is inclusive. I will strive to
do this and I ask that you do the same. If I can do anything to make the classroom a better
learning environment for you, please let me know.

\sphinxAtStartPar
\sphinxstylestrong{If you observe or experience behaviors that violate our commitment to inclusivity,
please let me know as soon as possible.}

\sphinxAtStartPar
If I violate this principle, please let me know or please tell the undergraduate department chair, Stuart Tessmer (\sphinxhref{mailto:tessmer@pa.msu.edu}{tessmer@pa.msu.edu}), who I have informed to tell me about any such incidents without conveying student information to me.


\subsection{Comments on preparation:}
\label{\detokenize{content/0_course/environment:comments-on-preparation}}
\sphinxAtStartPar
Physics 415 covers material you might have seen before. Many of the topics
stem from a wide variety of physics courses you might have already taken. But, we might be applying them at a higher level of conceptual and mathematical sophistication.

\sphinxAtStartPar
Therefore you should expect:
\begin{itemize}
\item {} 
\sphinxAtStartPar
a large amount of material to review and digest.

\item {} 
\sphinxAtStartPar
no recitations, and few examples covered in lecture. Most of the learning will be done through projects and questions you and your group mates raise.

\item {} 
\sphinxAtStartPar
long, hard problems that usually cannot be completed by one individual alone.

\item {} 
\sphinxAtStartPar
challenging projects.

\item {} 
\sphinxAtStartPar
to learn more about being a physicist that you have in another class (I hope!).

\end{itemize}

\sphinxAtStartPar
Physics 415 is a challenging, upper‐division physics course. Unlike more introductory courses, you are fully responsible for your own learning. In particular, you control the pace of the course by asking questions in class. I tend to speak quickly, and questions are important to slow down. This means that if you don’t understand something, it is your responsibility to ask questions. Attending class and the help sessions gives you an opportunity to ask questions. I am here to help you as much as possible, but I need your questions to know what you don’t understand.

\sphinxAtStartPar
Physics 415 covers some of the most important physics and mathematical methods in the field. Your reward for the hard work and effort will be learning important and elegant material that you will use over and over as a physics major. Here is what I have experienced, and heard from
other faculty teaching upper division physics in the past:
\begin{itemize}
\item {} 
\sphinxAtStartPar
most students reported spending a minimum of 10 hours per week on the
homework (!!)

\item {} 
\sphinxAtStartPar
students who didn’t attend the help sessions
often did poorly in the class.

\item {} 
\sphinxAtStartPar
students reported learning a tremendous amount in this class.

\end{itemize}

\sphinxAtStartPar
\sphinxstylestrong{The course topics that we will cover in Physics 415 are among the
greatest intellectual achievements of humans. Don’t be surprised if you
have to think hard and work hard to master the material.}

\sphinxstepscope


\section{Resources}
\label{\detokenize{content/0_course/resources:resources}}\label{\detokenize{content/0_course/resources::doc}}

\subsection{Confidentiality and Mandatory Reporting}
\label{\detokenize{content/0_course/resources:confidentiality-and-mandatory-reporting}}
\sphinxAtStartPar
College students often experience issues that may interfere with academic success such as academic stress, sleep problems, juggling responsibiities, life events, relationship concerns, or feelings of anxiety, hopelessness, or depression.
As your instructor, one of my responsibilities is to help create a safe learning environment and to support you through these situations and experiences.
I also have a mandatory reporting responsibility related to my role as a University employee.
It is my goal that you feel able to share information related to your life experiences in classroom
discussions, in written work, and in one\sphinxhyphen{}on\sphinxhyphen{}one meetings.
I will seek to keep information you share private to the greatest extent possible.
However, under Title IX, I am required to share information regarding sexual misconduct, relationship violence, or information
about criminal activity on MSU’s campus with the University including the Office of Institutional Equity (OIE).

\sphinxAtStartPar
\sphinxstylestrong{Students may speak to someone confidentially by contacting MSU Counseling and Psychiatric Service (CAPS) (\sphinxhref{http://caps.msu.edu}{caps.msu.edu}, 517\sphinxhyphen{}355\sphinxhyphen{}8270), MSU’s 24\sphinxhyphen{}hour Sexual Assault Crisis Line (\sphinxhref{http://endrape.msu.edu}{endrape.msu.edu}, 517\sphinxhyphen{}372\sphinxhyphen{}6666), or Olin Health Center (\sphinxhref{http://olin.msu.edu}{olin.msu.edu}, 517\sphinxhyphen{}884\sphinxhyphen{}6546).}


\subsection{Spartan Code of Honor Academic Pledge}
\label{\detokenize{content/0_course/resources:spartan-code-of-honor-academic-pledge}}
\sphinxAtStartPar
As a Spartan, I will strive to uphold values of the highest ethical standard. I will practice honesty in my work, foster honesty in my peers, and take pride in knowing that honor is worth more than grades. I will carry these values beyond my time as a student at Michigan State University, continuing the endeavor to build personal integrity in all that I do.


\subsection{Handling Emergency Situations}
\label{\detokenize{content/0_course/resources:handling-emergency-situations}}
\sphinxAtStartPar
\sphinxstyleemphasis{In the event of an emergency arising within the classroom, Prof. Caballero will notify you of what actions that may be required to ensure your safety. It is the responsibility of each student to understand the evacuation, “shelter\sphinxhyphen{}in\sphinxhyphen{}place,” and “secure\sphinxhyphen{}in\sphinxhyphen{}place” guidelines posted in each facility and to act in a safe manner. You are allowed to maintain cellular devices in a silent mode during this course, in order to receive emergency SMS text, phone or email messages distributed by the university. When anyone receives such a notification or observes an emergency situation, they should immediately bring it to the attention of Prof. Caballero in a way that causes the least disruption. If an evacuation is ordered, please ensure that you do it in a safe manner and facilitate those around you that may not otherwise be able to safely leave. When these orders are given, you do have the right as a member of this community to follow that order. Also, if a shelter\sphinxhyphen{}in\sphinxhyphen{}place or secure\sphinxhyphen{}in\sphinxhyphen{}place is ordered, please seek areas of refuge that are safe depending on the emergency encountered and provide assistance if it is advisable to do so.}

\sphinxstepscope


\chapter{Reading Questions Form}
\label{\detokenize{content/0_course/reading_questions:reading-questions-form}}\label{\detokenize{content/0_course/reading_questions::doc}}


\sphinxstepscope


\chapter{1 \sphinxhyphen{} What is Mathematical Modeling?}
\label{\detokenize{content/1_modeling/what_is_modeling:what-is-mathematical-modeling}}\label{\detokenize{content/1_modeling/what_is_modeling::doc}}
\sphinxAtStartPar
Nature reveals itself to us through interactions. We can tell from observations that it is nature’s interactions that lead to its evolution. How nature is changing and predicting how it will change in the future is the work of science. In this work, we observe nature and its interactions to make models of those observations. We aim to predict and explain our observations of nature through this building of models.

\sphinxAtStartPar
In physics, our goals are typically to explain and predict observations of physical phenomenon. Here, we focus ourselves to those canonical things that physicists concern themselves with: motion, fields, waves, atoms, nuclei, and so on.

\sphinxstepscope


\section{In\sphinxhyphen{}Class Activity: 01 \sphinxhyphen{} What is a Model?}
\label{\detokenize{content/1_modeling/activity-what_is_a_model:in-class-activity-01-what-is-a-model}}\label{\detokenize{content/1_modeling/activity-what_is_a_model::doc}}

\subsection{Notes}
\label{\detokenize{content/1_modeling/activity-what_is_a_model:notes}}\begin{itemize}
\item {} 
\sphinxAtStartPar
\sphinxstyleemphasis{Class Meeting:} 01 Sept 22

\item {} 
\sphinxAtStartPar
\sphinxstyleemphasis{Prior Reading:} None necessary

\end{itemize}


\subsection{Models and modeling}
\label{\detokenize{content/1_modeling/activity-what_is_a_model:models-and-modeling}}
\sphinxAtStartPar
This course focuses on common physical models and how we use mathematics and computing tools to investigate these models. But, we need to agree on an mutual understanding of a model. In this activity, you will be working in a group to define aspects of a good model for the physical sciences.

\sphinxAtStartPar
There are plenty of ideas on the internet, I can even \sphinxhref{https://bfy.tw/TUdA}{provide you a link} with plenty of information. But, the point of this activity is for us to define a good model now (given all your prior experiences with models and modeling), and to reflect on that definition as you develop expertise with models and modeling (through your learning in this class).


\subsection{What is a model?}
\label{\detokenize{content/1_modeling/activity-what_is_a_model:what-is-a-model}}
\sphinxAtStartPar
We are going to start with this short video made by Geoscientist John Aiken when he was a graduate student at Georgia Tech.

\begin{sphinxuseclass}{cell}\begin{sphinxVerbatimInput}

\begin{sphinxuseclass}{cell_input}
\begin{sphinxVerbatim}[commandchars=\\\{\}]
\PYG{k+kn}{from} \PYG{n+nn}{IPython}\PYG{n+nn}{.}\PYG{n+nn}{display} \PYG{k+kn}{import} \PYG{n}{YouTubeVideo}
\PYG{n}{YouTubeVideo}\PYG{p}{(}\PYG{l+s+s2}{\PYGZdq{}}\PYG{l+s+s2}{dkTncoPqo5Y}\PYG{l+s+s2}{\PYGZdq{}}\PYG{p}{,} \PYG{n}{width} \PYG{o}{=} \PYG{l+m+mi}{800}\PYG{p}{,} \PYG{n}{height} \PYG{o}{=} \PYG{l+m+mi}{600}\PYG{p}{)}
\end{sphinxVerbatim}

\end{sphinxuseclass}\end{sphinxVerbatimInput}
\begin{sphinxVerbatimOutput}

\begin{sphinxuseclass}{cell_output}
\noindent\sphinxincludegraphics{{activity-what_is_a_model_2_0}.jpg}

\end{sphinxuseclass}\end{sphinxVerbatimOutput}

\end{sphinxuseclass}

\subsubsection{History and Philosophy of Science}
\label{\detokenize{content/1_modeling/activity-what_is_a_model:history-and-philosophy-of-science}}
\sphinxAtStartPar
If you would like to dive deeper into models and modeling, there’s excellent work in history and philosophy of science. The field studies how science develops knowledge, practice, culture, and so on. It studies important events and provides critical information on important and, often, overlooked folks who do science. For example, historian and gender studies professor \sphinxhref{https://en.wikipedia.org/wiki/Sharon\_Traweek}{Sharon Traweek} studies the high energy physics field. Her book, \sphinxhref{https://en.wikipedia.org/wiki/Beamtimes\_and\_Lifetimes}{Beamtimes and Lifetimes: The World of High Energy Physicists} {[}\hyperlink{cite.content/X_additional_pages/references-page:id4}{Traweek, 2009}{]} is excellent.

\begin{sphinxadmonition}{note}{Dame Nancy Cartwright (philosopher of science)}

\sphinxAtStartPar
One of the more interesting scholars in \sphinxhref{https://en.wikipedia.org/wiki/Nancy\_Cartwright\_(philosopher)}{Dame Nancy Cartwright} who wrote a lot about the ‘practice of science.’ Her philosophical work informed many of the innovations in physics and broader science education – including many science courses at MSU.

\sphinxAtStartPar
Her writing is very interesting, but the style of writing can be a challenge to read. This is the nature of academic writing in different disciplines. Her book called “How The Laws of Physics Lie” {[}\hyperlink{cite.content/X_additional_pages/references-page:id5}{Cartwright, 1983}{]} is worth a read. Here’s a link to the \sphinxhref{http://www.generativescience.org/papers/nature/Cartwright-\_1983.pdf}{first chapter}.
\end{sphinxadmonition}


\subsection{What has been your experience with models?}
\label{\detokenize{content/1_modeling/activity-what_is_a_model:what-has-been-your-experience-with-models}}
\sphinxAtStartPar
Take 2\sphinxhyphen{}3 minutes to think about your prior physics class.
\begin{itemize}
\item {} 
\sphinxAtStartPar
What models have you used? What makes that a model?

\item {} 
\sphinxAtStartPar
What modeling have you done? What makes that modeling?

\end{itemize}

\begin{sphinxuseclass}{cell}\begin{sphinxVerbatimInput}

\begin{sphinxuseclass}{cell_input}
\begin{sphinxVerbatim}[commandchars=\\\{\}]
\PYG{c+c1}{\PYGZsh{}\PYGZsh{} Take notes for yourself here}

\PYG{c+c1}{\PYGZsh{} Idea 1}
\PYG{c+c1}{\PYGZsh{} Idea 2}
\end{sphinxVerbatim}

\end{sphinxuseclass}\end{sphinxVerbatimInput}

\end{sphinxuseclass}
\sphinxAtStartPar
After we have discussed the above questions and generated a long list of experiences, you can move onto the next set of questions below.

\begin{sphinxuseclass}{toggle}
\sphinxAtStartPar
Take 2\sphinxhyphen{}3 minutes to think about your prior work with models and modeling?
\begin{itemize}
\item {} 
\sphinxAtStartPar
What made a model good or not so good?

\item {} 
\sphinxAtStartPar
What kinds of things could you do to make a better model?

\end{itemize}

\end{sphinxuseclass}
\begin{sphinxuseclass}{cell}\begin{sphinxVerbatimInput}

\begin{sphinxuseclass}{cell_input}
\begin{sphinxVerbatim}[commandchars=\\\{\}]
\PYG{c+c1}{\PYGZsh{}\PYGZsh{} Take notes for yourself here}

\PYG{c+c1}{\PYGZsh{} Idea 1}
\PYG{c+c1}{\PYGZsh{} Idea 2}
\end{sphinxVerbatim}

\end{sphinxuseclass}\end{sphinxVerbatimInput}

\end{sphinxuseclass}
\sphinxAtStartPar
After we have discussed the above questions and generated a consensus list of good model features, you can move onto the next set of questions below.

\begin{sphinxuseclass}{toggle}
\sphinxAtStartPar
Take 2\sphinxhyphen{}3 minutes and write down all the mathematical and/or computing models you have experienced, heard about, seen, or are otherwise familiar.

\end{sphinxuseclass}
\begin{sphinxuseclass}{cell}\begin{sphinxVerbatimInput}

\begin{sphinxuseclass}{cell_input}
\begin{sphinxVerbatim}[commandchars=\\\{\}]
\PYG{c+c1}{\PYGZsh{}\PYGZsh{} Take notes for yourself here (use comments)}

\PYG{c+c1}{\PYGZsh{} Model 1}
\PYG{c+c1}{\PYGZsh{} Model 2}
\end{sphinxVerbatim}

\end{sphinxuseclass}\end{sphinxVerbatimInput}

\end{sphinxuseclass}
\sphinxstepscope


\section{Notes: 01 \sphinxhyphen{} What is a Model?}
\label{\detokenize{content/1_modeling/notes-what_is_a_model:notes-01-what-is-a-model}}\label{\detokenize{content/1_modeling/notes-what_is_a_model::doc}}
\sphinxAtStartPar
These is the summary of the PHY 415 class period on 01 Sept 22. If I missed anything, \sphinxhref{https://phy415fall2022.slack.com/}{send me an message} on Slack and I will add it here. \sphinxhyphen{}DC


\subsection{Modeling Discussion}
\label{\detokenize{content/1_modeling/notes-what_is_a_model:modeling-discussion}}
\sphinxAtStartPar
We discussed models and modeling extensively today to start to form a functional definition of both. We talked about what the general idea of models and modeling can encompass. Some of the classes major ideas were:
\begin{itemize}
\item {} 
\sphinxAtStartPar
A model takes \sphinxstylestrong{BIG} things and makes them \sphinxstyleemphasis{smaller} things (approximations and assumptions)

\item {} 
\sphinxAtStartPar
A model is only as good as it fits nature/reality/experiment

\item {} 
\sphinxAtStartPar
A model is not useful unless it does something we intended it

\item {} 
\sphinxAtStartPar
The best models destroy themselves (i.e., are pushed to their limits and need to be replaced/redeveloped)

\end{itemize}


\subsubsection{Mathematical Models in Physics}
\label{\detokenize{content/1_modeling/notes-what_is_a_model:mathematical-models-in-physics}}
\sphinxAtStartPar
We narrows our discussion to mathematical models, the focus of this class, and talked about the variety of models the class has used. We had a couple general ideas. A \sphinxstylestrong{mathematical model} (in physics) is:
\begin{itemize}
\item {} 
\sphinxAtStartPar
an equation that has a physics idea behind it

\item {} 
\sphinxAtStartPar
the order that you use different mathematical methods to investigate a physics problem

\end{itemize}

\sphinxAtStartPar
In this discussion, the class brought up several models they had experienced in the past; many of them were from mechanics contexts. And a couple were outside of physics models.

\sphinxAtStartPar
\sphinxstylestrong{Physics}
\begin{itemize}
\item {} 
\sphinxAtStartPar
Hooke’s Law

\item {} 
\sphinxAtStartPar
Differential equations that describe physical systems
\begin{itemize}
\item {} 
\sphinxAtStartPar
example: the Schrodinger equation

\end{itemize}

\item {} 
\sphinxAtStartPar
Taylor expansions

\item {} 
\sphinxAtStartPar
Newton’s Laws
\begin{itemize}
\item {} 
\sphinxAtStartPar
Kepler’s Laws

\item {} 
\sphinxAtStartPar
Newton’s Law of Gravitation

\end{itemize}

\end{itemize}

\sphinxAtStartPar
\sphinxstylestrong{Outside of Physics}
\begin{itemize}
\item {} 
\sphinxAtStartPar
\sphinxhref{https://en.wikipedia.org/wiki/Zipf\%27s\_law}{Zipf’s Law} (linguistics)

\item {} 
\sphinxAtStartPar
\sphinxhref{https://en.wikipedia.org/wiki/Benford\%27s\_law}{Benford’s Law} (random numbers)

\end{itemize}


\subsubsection{Features of a good mathematical model}
\label{\detokenize{content/1_modeling/notes-what_is_a_model:features-of-a-good-mathematical-model}}
\sphinxAtStartPar
We then discussed the features that make up a good model in physics. Ideas included aspects of the model and the modeling process as well as social aspects of engaging in science. The list below is from the class’s discussion, where the highlighted words are starting to form the details of our class’s project rubric.

\sphinxAtStartPar
\sphinxstylestrong{A good model:}
\begin{itemize}
\item {} 
\sphinxAtStartPar
depends on its intent and how well it does what it we intend it to do
\begin{itemize}
\item {} 
\sphinxAtStartPar
e.g., a model of a galaxy is not useful for modeling the solar system

\end{itemize}

\item {} 
\sphinxAtStartPar
predicts or explains something about a physical system \sphinxhyphen{} \sphinxstylestrong{predictability}

\item {} 
\sphinxAtStartPar
gives everyone the same results with the same input \sphinxhyphen{} \sphinxstylestrong{reproducibility}

\item {} 
\sphinxAtStartPar
balances simplicity and applicability \sphinxhyphen{} \sphinxstylestrong{parsimonius}

\item {} 
\sphinxAtStartPar
cannot be biased towards the data (experiment, etc.)
\begin{itemize}
\item {} 
\sphinxAtStartPar
avoid manipulation of the model to fit data

\item {} 
\sphinxAtStartPar
overfitting data

\end{itemize}

\item {} 
\sphinxAtStartPar
recognizes \sphinxstylestrong{limitations} and presents its \sphinxstylestrong{assumptions}

\item {} 
\sphinxAtStartPar
uses input from outside sources (other theories, data, experiments, etc) to confirm/evaluate \sphinxhyphen{} \sphinxstylestrong{validation}

\item {} 
\sphinxAtStartPar
can be adjusted for different conditions \sphinxhyphen{} \sphinxstylestrong{investigatory}

\item {} 
\sphinxAtStartPar
is \sphinxstylestrong{testable} in some way

\item {} 
\sphinxAtStartPar
needs to have its results and processes \sphinxstylestrong{communicated clearly}

\end{itemize}

\sphinxstepscope


\chapter{2 \sphinxhyphen{} Oscillations}
\label{\detokenize{content/2_oscillations/readings-oscillators:oscillations}}\label{\detokenize{content/2_oscillations/readings-oscillators::doc}}
\sphinxAtStartPar
Oscillations are ubiquitous in nature. We see it in the motion of springs and pendulums, clearly. But orbital motion of the planets and stellar objects are large scale (both time and space) oscillations. While not obvious, weather patterns, the stock market, and other nonlinear and complex systems exhibit oscillations. Particle oscillations in quantum mechanics are some of the most interesting new physics we have discovered. Check out this short video on the oscillation of neutrinos.

\begin{sphinxuseclass}{cell}\begin{sphinxVerbatimInput}

\begin{sphinxuseclass}{cell_input}
\begin{sphinxVerbatim}[commandchars=\\\{\}]
\PYG{k+kn}{from} \PYG{n+nn}{IPython}\PYG{n+nn}{.}\PYG{n+nn}{display} \PYG{k+kn}{import} \PYG{n}{YouTubeVideo}
\PYG{n}{YouTubeVideo}\PYG{p}{(}\PYG{l+s+s2}{\PYGZdq{}}\PYG{l+s+s2}{7fgKBJDMO54}\PYG{l+s+s2}{\PYGZdq{}}\PYG{p}{,} \PYG{l+m+mi}{800}\PYG{p}{,} \PYG{l+m+mi}{600}\PYG{p}{)}
\end{sphinxVerbatim}

\end{sphinxuseclass}\end{sphinxVerbatimInput}
\begin{sphinxVerbatimOutput}

\begin{sphinxuseclass}{cell_output}
\noindent\sphinxincludegraphics{{readings-oscillators_1_0}.jpg}

\end{sphinxuseclass}\end{sphinxVerbatimOutput}

\end{sphinxuseclass}

\section{Neutrino Oscillations}
\label{\detokenize{content/2_oscillations/readings-oscillators:neutrino-oscillations}}
\sphinxAtStartPar
These \sphinxhref{https://en.wikipedia.org/wiki/Neutrino\_oscillation}{neutrino oscillations} are incredibly important to the future of high energy and particle physics. Because of their observed oscillation, it is clear the neutrino is not a massless particle, but how much mass is an open question. The work that lead to this discovery was lead by Takaaki Kajita and Arthur McDonald who jointly were awarded the \sphinxhref{https://www.nobelprize.org/prizes/physics/2015/press-release/}{2015 Nobel Prize in physics}.

\begin{sphinxadmonition}{note}{Current Research at Michigan State}

\sphinxAtStartPar
The Physics and Astronomy Department at MSU has a research effort devoted to neutrinos and neutrino oscillations. Some of this work is lead by \sphinxhref{https://www.dunescience.org/facesofdune/kendall-mahn/}{Prof. Kendall Mahn} who works on the \sphinxhref{https://www.dunescience.org/}{Deep Underground Neutrino Experiment (DUNE)}. The work of her and her team have the potential to \sphinxhref{https://www.scientificamerican.com/article/is-the-standard-model-of-physics-now-broken/}{break standard model physics} and move us into the realm of new particle physics.
\end{sphinxadmonition}


\section{The Simple Harmonic Oscillator}
\label{\detokenize{content/2_oscillations/readings-oscillators:the-simple-harmonic-oscillator}}
\sphinxAtStartPar
One of the central models of physics is the simple harmonic oscillator; both beloved and ridiculed, it’s hard to deny how often it gets used. At issue is that nature tends towards some equilibrium. Most systems left alone will relax to some equilibrium. Typically, the process minimizes some quantity (e.g., energy) and thus we seek energy wells. These wells may have a complex form, but only depend on position. In 1d, the potential energy (also called the “configuration energy”) is:
\begin{equation*}
\begin{split}U = U(x)\end{split}
\end{equation*}
\sphinxAtStartPar
A potential energy extrema will have a vanishing first derivative (\(dU/dx=0\)). An energy minimum will have a positive concavity (\(d^2U/dx^2>0\)).

\sphinxAtStartPar
Consider a strange potential energy function:
\begin{equation*}
\begin{split}U(x)= \left[5\sin(3x)+(x+0.5)^2\right]\;e^{(-\dfrac{x}{10})}\end{split}
\end{equation*}
\sphinxAtStartPar
It is plotted below. Notice the many local potential energy minima. Motion around each of those points can be modeled as a harmonic oscillator.

\begin{sphinxuseclass}{cell}\begin{sphinxVerbatimInput}

\begin{sphinxuseclass}{cell_input}
\begin{sphinxVerbatim}[commandchars=\\\{\}]
\PYG{k+kn}{import} \PYG{n+nn}{numpy} \PYG{k}{as} \PYG{n+nn}{np}
\PYG{k+kn}{import} \PYG{n+nn}{matplotlib}\PYG{n+nn}{.}\PYG{n+nn}{pyplot} \PYG{k}{as} \PYG{n+nn}{plt}

\PYG{n}{x} \PYG{o}{=} \PYG{n}{np}\PYG{o}{.}\PYG{n}{arange}\PYG{p}{(}\PYG{o}{\PYGZhy{}}\PYG{l+m+mi}{10}\PYG{p}{,}\PYG{l+m+mi}{10}\PYG{p}{,}\PYG{l+m+mf}{0.01}\PYG{p}{)}
\PYG{n}{U} \PYG{o}{=} \PYG{p}{(}\PYG{l+m+mi}{5}\PYG{o}{*}\PYG{n}{np}\PYG{o}{.}\PYG{n}{sin}\PYG{p}{(}\PYG{l+m+mi}{3}\PYG{o}{*}\PYG{n}{x}\PYG{p}{)}\PYG{o}{+}\PYG{p}{(}\PYG{n}{x}\PYG{o}{+}\PYG{l+m+mf}{0.5}\PYG{p}{)}\PYG{o}{*}\PYG{o}{*}\PYG{l+m+mi}{2}\PYG{p}{)}\PYG{o}{*}\PYG{n}{np}\PYG{o}{.}\PYG{n}{exp}\PYG{p}{(}\PYG{o}{\PYGZhy{}}\PYG{l+m+mi}{1}\PYG{o}{/}\PYG{l+m+mi}{10}\PYG{o}{*}\PYG{n}{x}\PYG{p}{)}

\PYG{n}{plt}\PYG{o}{.}\PYG{n}{plot}\PYG{p}{(}\PYG{n}{x}\PYG{p}{,}\PYG{n}{U}\PYG{p}{)}
\end{sphinxVerbatim}

\end{sphinxuseclass}\end{sphinxVerbatimInput}
\begin{sphinxVerbatimOutput}

\begin{sphinxuseclass}{cell_output}
\begin{sphinxVerbatim}[commandchars=\\\{\}]
[\PYGZlt{}matplotlib.lines.Line2D at 0x1170c9ed0\PYGZgt{}]
\end{sphinxVerbatim}

\noindent\sphinxincludegraphics{{readings-oscillators_4_1}.png}

\end{sphinxuseclass}\end{sphinxVerbatimOutput}

\end{sphinxuseclass}

\subsection{How are we sure we can model each local minimum as an SHO?}
\label{\detokenize{content/2_oscillations/readings-oscillators:how-are-we-sure-we-can-model-each-local-minimum-as-an-sho}}
\sphinxAtStartPar
Remember that we are modeling energy minima, so that \(dU/dx=0\) at the point we care about (the local energy minimum). So if we expand the potential in a Taylor series around any minimum energy point (\(x_{localMinEnergy}\)), we find the first non\sphinxhyphen{}constant contribution is quadratic,
\begin{equation*}
\begin{split}U(x) \approx U(x)|_{x_{localMinEnergy}} + \dfrac{dU}{dx}|_{x_{localMinEnergy}}(x-x_{localMinEnergy}) +\end{split}
\end{equation*}\begin{equation*}
\begin{split}\dfrac{1}{2}\dfrac{d^2U}{dx^2}|_{x_{localMinEnergy}}(x-x_{localMinEnergy})^2 + O((x-x_{localMinEnergy})^3)\end{split}
\end{equation*}
\sphinxAtStartPar
But, by definition, at a potential energy minimum
\begin{equation*}
\begin{split}\dfrac{dU}{dx}|_{x_{localMinEnergy}}=0\end{split}
\end{equation*}
\sphinxAtStartPar
We neglect higher order terms, which arguably limits the extent we can predict. That is the space around the potential energy minimum that we can predict with good accuracy depends very much on how much the energy changes near the minimum and how well it can be approximated by a parabola. But, we are left with:
\begin{equation*}
\begin{split}U(x) \approx U(x)|_{x_{localMinEnergy}} + \dfrac{1}{2}\dfrac{d^2U}{dx^2}|_{x_{localMinEnergy}}(x-x_{localMinEnergy})^2\end{split}
\end{equation*}
\sphinxAtStartPar
which is an SHO potential energy function.


\subsection{The SHO mathematical model}
\label{\detokenize{content/2_oscillations/readings-oscillators:the-sho-mathematical-model}}
\sphinxAtStartPar
From this potential energy, we find that a reasonable force model (\(F = -dU/dx\)) for a 1D spring system is given by:
\begin{equation*}
\begin{split}F_{spring} = -k |x-x_0|\end{split}
\end{equation*}
\sphinxAtStartPar
where \(k\) is the spring constant for the spring and \(x_0\) is the equilibrium location. As we know from Newton’s second law, if we can argue that this force is the only force or the dominant behavior we are trying to model in a mechanical situation, we have the following ordinary differential equation:
\begin{equation*}
\begin{split}F_{net} = F_{spring}\end{split}
\end{equation*}\begin{equation*}
\begin{split}m\ddot{x} = -k|x-x_0|\end{split}
\end{equation*}
\sphinxAtStartPar
Without much loss, we can recast the problem in terms of the distance from equilibrium
\begin{equation*}
\begin{split}s=|x-x_0|\end{split}
\end{equation*}
\sphinxAtStartPar
\sphinxstyleemphasis{As you will learn this is not a problem for vertically hanging springs near the surface of the Earth because the gravitational force is constant and depends on the mass of the hanging weight.}

\begin{sphinxadmonition}{note}{The SHO mathematical model}
\begin{equation*}
\begin{split}m\ddot{s} = -ks\end{split}
\end{equation*}\end{sphinxadmonition}


\subsection{Finding the general solution}
\label{\detokenize{content/2_oscillations/readings-oscillators:finding-the-general-solution}}
\sphinxAtStartPar
We can solve this using our typical approach to 2nd order linear ODEs; guess the form of the solution and check it. \sphinxstyleemphasis{You might have done this in the past with functions of \(\sin\) and \(\cos\). Here, we will introduce the complex form of the solution as a compliment.}
\begin{equation*}
\begin{split}\ddot{s} = -\frac{k}{m}s\end{split}
\end{equation*}
\sphinxAtStartPar
Assume: \(s(t) = A\exp(i\omega t)\) As we will learn this form is a good guess for linear ODEs. You might be worried about the \(i\), which is a complex number \(i = \sqrt{-1}\). That’s ok, for now, let’s use it. Then we will see how it connects.

\sphinxAtStartPar
Putting in the guessed solution (“the ansatz”) gives:
\begin{equation*}
\begin{split}\ddot{s} = -\omega^2\;A\exp(i\omega t)\end{split}
\end{equation*}
\sphinxAtStartPar
So that,
\begin{equation*}
\begin{split}-\omega^2\;A\exp(i\omega t) = -\frac{k}{m}A\exp(i\omega t)\end{split}
\end{equation*}
\sphinxAtStartPar
This solution is consistent with the model (i.e., solves the differential equation) if we set \(\omega\) and notice it is exclusively positive:
\begin{equation*}
\begin{split}\omega^2 = \frac{k}{m} > 0\end{split}
\end{equation*}
\sphinxAtStartPar
Thus, \(s(t) = A\exp(i\sqrt{\frac{k}{m}} t)\) is \sphinxstylestrong{a solution} to this problem. Notice we have to set \(A\) using initial conditions for the problem. This is called a \sphinxstylestrong{initial value problem} in mathematics.


\subsubsection{What about that pesky imaginary part?}
\label{\detokenize{content/2_oscillations/readings-oscillators:what-about-that-pesky-imaginary-part}}
\sphinxAtStartPar
Our complex formulation is consistent with the \(\sin\) and \(\cos\) approach you have used in the past. You might have heard about the Euler formulation; it is one of the most important relationships in mathematics {[}\hyperlink{cite.content/X_additional_pages/references-page:id2}{Nahin, 2011}, \hyperlink{cite.content/X_additional_pages/references-page:id3}{Stipp, 2017}{]}. We will get into it more later. It appears below:

\begin{sphinxadmonition}{note}{Leonhard Euler’s famous formula}
\begin{equation*}
\begin{split}\exp{i\theta} = \cos{\theta} + i \sin{\theta}\end{split}
\end{equation*}\end{sphinxadmonition}

\sphinxAtStartPar
Thus, we know that our solution gives rise to real sinusoidal solutions of the form:
\begin{equation*}
\begin{split}s(t) = A\left(cos{\sqrt{\frac{k}{m}} t} + i\sin{\sqrt{\frac{k}{m}} t}\right)\end{split}
\end{equation*}
\sphinxAtStartPar
This probably doesn’t look quite right…there’s an \(i\) in the solution for real physical motion! We will get to that, but we need the initial conditions to do so. More importantly, there’s only one free parameter in our current solution, \(A\). Given that this equation is \sphinxstylestrong{a solution, is it the only solution?}


\subsection{The order of your differential equation matters}
\label{\detokenize{content/2_oscillations/readings-oscillators:the-order-of-your-differential-equation-matters}}
\sphinxAtStartPar
You probably already know that our general solution for the SHO is:
\begin{equation*}
\begin{split}s(t) = A\exp(i\sqrt{\frac{k}{m}} t) + B\exp(-i\sqrt{\frac{k}{m}} t)\end{split}
\end{equation*}
\sphinxAtStartPar
Or written in a way that is more often presented:
\begin{equation*}
\begin{split}s(t) = A'\cos(\sqrt{\frac{k}{m}} t) + B'\sin(\sqrt{\frac{k}{m}} t)\end{split}
\end{equation*}
\sphinxAtStartPar
where we have made clear that the free parameters \(A\) and \(B\) are not equal to \(A'\) and \(B'\), respectively. Regardless, notice there’s two free parameters in these formula. That is because we are solving a \sphinxstylestrong{two dimensional} linear ODE. Because there’s two derivatives on \(s\) (i.e., \(\ddot{s}\)), we expect to have two free parameters that are set by the intial conditions. This is characteristic of an \sphinxstylestrong{initial value problem}.

\begin{sphinxadmonition}{note}{Important Tip}

\sphinxAtStartPar
The number of free parameters that are set by problem conditions in an initial value problem is determined by the order of the differential equation.

\sphinxAtStartPar
Second order? Two conditions needed. Two free parameters to determine. Fourth order? Four conditions needed. Four free parameters to determine.
\end{sphinxadmonition}


\subsection{More Resources}
\label{\detokenize{content/2_oscillations/readings-oscillators:more-resources}}
\sphinxAtStartPar
If you are feeling less comfortable with the harmonic oscillator or it’s been some time, I think Prof. Ten\sphinxhyphen{}Jie Lee’s lecture is very good. It’s long (a full 50 minutes), but here’s some key points:
\begin{itemize}
\item {} 
\sphinxAtStartPar
About 10 minutes in, he derives the full solution (we will not do this)

\item {} 
\sphinxAtStartPar
About 17 minutes in, he shows how to get particular solutions with initial conditions (we will do this)

\item {} 
\sphinxAtStartPar
About 26 minutes in, he completes an example where he shows how to Taylor approximate a potential to use the SHO model (we will do something like this)

\item {} 
\sphinxAtStartPar
About 40 minutes in, he discusses the different forms of the solutions (we will do this)

\item {} 
\sphinxAtStartPar
About 43 minutes in, he brings in complex analysis (we will do this)

\end{itemize}

\begin{sphinxuseclass}{cell}\begin{sphinxVerbatimInput}

\begin{sphinxuseclass}{cell_input}
\begin{sphinxVerbatim}[commandchars=\\\{\}]
\PYG{n}{YouTubeVideo}\PYG{p}{(}\PYG{l+s+s2}{\PYGZdq{}}\PYG{l+s+s2}{4ysFC9vd3GE}\PYG{l+s+s2}{\PYGZdq{}}\PYG{p}{,} \PYG{l+m+mi}{800}\PYG{p}{,} \PYG{l+m+mi}{600}\PYG{p}{)}
\end{sphinxVerbatim}

\end{sphinxuseclass}\end{sphinxVerbatimInput}
\begin{sphinxVerbatimOutput}

\begin{sphinxuseclass}{cell_output}
\noindent\sphinxincludegraphics{{readings-oscillators_9_0}.jpg}

\end{sphinxuseclass}\end{sphinxVerbatimOutput}

\end{sphinxuseclass}
\sphinxstepscope


\section{Activity: Investigating the Simple Harmonic Oscillator}
\label{\detokenize{content/2_oscillations/activity-SHO_and_dynamical_systems:activity-investigating-the-simple-harmonic-oscillator}}\label{\detokenize{content/2_oscillations/activity-SHO_and_dynamical_systems::doc}}

\subsection{Reminders of the SHO and Introduction to Dynamical Systems}
\label{\detokenize{content/2_oscillations/activity-SHO_and_dynamical_systems:reminders-of-the-sho-and-introduction-to-dynamical-systems}}
\sphinxAtStartPar
Look for \sphinxstylestrong{✅ Do this}


\subsubsection{The SHO mathematical model}
\label{\detokenize{content/2_oscillations/activity-SHO_and_dynamical_systems:the-sho-mathematical-model}}
\sphinxAtStartPar
We have seen for the canonical one\sphinxhyphen{}dimensional harmonic oscillator is described by the following \sphinxstylestrong{2nd order linear} differential equation:
\begin{equation*}
\begin{split}m\ddot{s} = -k{s}\end{split}
\end{equation*}
\sphinxAtStartPar
where \(s\) represents the distance from the oscillators equilibrium position. It’s important to note that this model for a linear restoring force gives rise to at \sphinxstylestrong{econd order} differential equation. As you will see, there would be no ability for cycles or recurrent behavior without two dimensions (the two dimensions here are the position, \(s\), and velocity of the oscillator, \(\dot{s}\)). It is also important that the differential equation is \sphinxstylestrong{linear}.


\subsubsection{Linearity of differential equations}
\label{\detokenize{content/2_oscillations/activity-SHO_and_dynamical_systems:linearity-of-differential-equations}}
\sphinxAtStartPar
This is something that is easy to confuse because we use the term \sphinxstylestrong{linear} in different ways at different times. In many cases this prompts us to think of a line, so that we might think the only linear differential equations are ones where the variable of interest (in this case the position of the oscillator) appears only \sphinxstyleemphasis{linearly} in the equation.

\begin{sphinxadmonition}{warning}{Warning:}
\sphinxAtStartPar
So you might think that Nth order differential equations of the form:
\begin{equation*}
\begin{split}\dfrac{d^n x}{dt^n} = c\;x\end{split}
\end{equation*}
\sphinxAtStartPar
are the only ones that are linear. \sphinxstyleemphasis{Note for the harmonic oscillator \(n=2\).}
\end{sphinxadmonition}

\sphinxAtStartPar
\sphinxstylestrong{That is not the case and this is key.}

\sphinxAtStartPar
The linearity refers to the fact that we can write the differential equation as a \sphinxstylestrong{linear combination of the derivatives}.

\begin{sphinxadmonition}{tip}{Tip:}
\sphinxAtStartPar
So an \(N\)th linear differential equation (e.g., for position and time) can have the general form:
\begin{equation*}
\begin{split}\sum_{i=0}^N a_i(t)\dfrac{d^ix}{dt} = f(t)\end{split}
\end{equation*}
\sphinxAtStartPar
where \(\dfrac{d^ix}{dt}\) is the \(i\)th derivative of \(x\) (e.g., \(i=0\), \(\dfrac{d^0x}{dt}=x\); \(i=1\), \(\dfrac{d^0x}{dt}=\dfrac{dx}{dt}\)) The coefficients are allowed to depend explicitly on time (i.e., \(a_i(t)\); you have probably seen \sphinxstyleemphasis{constant coefficients} in the past (where \(\dfrac{da_i(t)}{dt}=0\), i.e., \(a_i(t) =\) constant). In addition, as you can explore later, the system can be driven by a function that depends only on time (e.g., some forcing function, \(f(t)\)).
\end{sphinxadmonition}


\subsubsection{Why does linearity matter?}
\label{\detokenize{content/2_oscillations/activity-SHO_and_dynamical_systems:why-does-linearity-matter}}
\sphinxAtStartPar
Because much of nature can be modeled with linear differential equations (or approximately so) and linear differential equations have really nice solution properties. *The solutions to linear differential equations are \sphinxstylestrong{holonomic functions}. This is just a fancy word for functions you know like polynomials, sine, cosine, exponentiale, and some special functions (e.g., Airy, Bessel, hypergeometric). But these solutions are really important because they are smooth as are their derivatives, so they have really nice properties.

\sphinxAtStartPar
In many cases, we can lean on strong properties dervied from linearity:
\begin{itemize}
\item {} 
\sphinxAtStartPar
\sphinxhref{https://en.wikipedia.org/wiki/Cauchy\%E2\%80\%93Kowalevski\_theorem}{Uniqueness}: If we find a solution to the differential equation and it satisfies the boundary/initial conditions, we are guaranteed that this is the only solution.

\item {} 
\sphinxAtStartPar
Linear combinations: The linear nature of the differential equation means linear combinations of general solutions are unchanged. We can build solutions from known general solutions.

\end{itemize}

\begin{sphinxadmonition}{note}{The Cauchy\sphinxhyphen{}Kowaleski theorem }

\sphinxAtStartPar
This theorem is typically called the Cauchy theorem (or the uniqueness theorem) for \sphinxhref{https://en.wikipedia.org/wiki/Augustin-Louis\_Cauchy}{Augustin Louis\sphinxhyphen{}Cachy}, a French mathematician who contributed much to the field of complex analysis. However, Cauchy only proved this theorem in a special case.  A Russian Mathematician, \sphinxhref{https://en.wikipedia.org/wiki/Sofya\_Kovalevskaya}{Sofya Kovalevskaya}, who was the first woman to earn a PhD in Mathematics (earned 1874, University of Göttingen), proved the theorem, in general, in her dissertation.

\sphinxAtStartPar
\sphinxstylestrong{So, it is really the “Kovalevskaya\sphinxhyphen{}Cauchy” theorem.}
\end{sphinxadmonition}


\subsubsection{General Solution to the SHO}
\label{\detokenize{content/2_oscillations/activity-SHO_and_dynamical_systems:general-solution-to-the-sho}}
\sphinxAtStartPar
We can start with the differential equation:
\begin{equation*}
\begin{split}\ddot{x} = -\omega\end{split}
\end{equation*}
\sphinxAtStartPar
where \(x\) is the position relative to equilibrium and \(\omega^2 = k/m\). We know this is a linear, second\sphinxhyphen{}order ODE, so we expect the solutions to be \sphinxhref{https://en.wikipedia.org/wiki/Holonomic\_function}{holonomic functions} \sphinxhyphen{} that is, many of the functions we have seen before. Moreover, we need something that when we take two derivatives, we get the same functions back.

\sphinxAtStartPar
\sphinxstyleemphasis{This is a key to suggest \(\sin\) and \(\cos\) as our proposed linear combination.}

\sphinxAtStartPar
Now, we have to decide how to do that. Because \sphinxstylestrong{ANY} linear combination is a good starting guess, but not all of them a reasonable guesses. How do we decide?


\paragraph{Discussion Question}
\label{\detokenize{content/2_oscillations/activity-SHO_and_dynamical_systems:discussion-question}}
\sphinxAtStartPar
\sphinxstylestrong{✅ Do this}

\sphinxAtStartPar
Below are several choices of potential general solution guesses, called “ansatz” in many books and classes. Which of them will work for us? How do you know? \sphinxstyleemphasis{Discuss with your neighbors} Demonstrate that one solution you chose works.
\begin{equation*}
\begin{split}x(t) = A\cos(\omega t)\end{split}
\end{equation*}\begin{equation*}
\begin{split}x(t) = B\sin(\omega t)\end{split}
\end{equation*}\begin{equation*}
\begin{split}x(t) = A\cos(\omega t)+B\sin(\omega t)\end{split}
\end{equation*}\begin{equation*}
\begin{split}x(t) = B\sin(\omega t+ \phi_B)\end{split}
\end{equation*}\begin{equation*}
\begin{split}x(t) = \sum_i^N A_i\cos(\omega_i t) + B_i\sin(\omega_i t)\end{split}
\end{equation*}
\sphinxAtStartPar
\sphinxstylestrong{What other general solution forms will work?}


\subsubsection{Particular Solutions}
\label{\detokenize{content/2_oscillations/activity-SHO_and_dynamical_systems:particular-solutions}}
\sphinxAtStartPar
Let’s pick two general solutions:
\begin{equation*}
\begin{split}x(t) = A\cos(\omega t) + B\sin(\omega t)\end{split}
\end{equation*}\begin{equation*}
\begin{split}x(t) = C\sin(\omega_t + \phi_C)\end{split}
\end{equation*}
\sphinxAtStartPar
\sphinxstylestrong{✅ Do this}
\begin{itemize}
\item {} 
\sphinxAtStartPar
For an SHO that is pulled to a point \(x_0\) and let go at \(t=0\), find the particular solutions using both general forms. \sphinxstylestrong{Compare and contrast these solutions with your neighbor.} What do you notice about them? How can they be made compatible with each other?

\item {} 
\sphinxAtStartPar
Make a plot (in this notebook) of both solutions for an SHO with frequency of 100Hz that starts at \(x_0 =\) 0.1m

\item {} 
\sphinxAtStartPar
Adjust the starting location and frequency, what do you notice?

\item {} 
\sphinxAtStartPar
\sphinxstylestrong{If you have time}, create another solution for an SHO that is put into motion with a velocity of +0.1m/s at \(x_0=0\) at \(t=0\).

\end{itemize}

\begin{sphinxuseclass}{cell}\begin{sphinxVerbatimInput}

\begin{sphinxuseclass}{cell_input}
\begin{sphinxVerbatim}[commandchars=\\\{\}]
\PYG{k+kn}{import} \PYG{n+nn}{numpy} \PYG{k}{as} \PYG{n+nn}{np}
\PYG{k+kn}{import} \PYG{n+nn}{matplotlib}\PYG{n+nn}{.}\PYG{n+nn}{pyplot} \PYG{k}{as} \PYG{n+nn}{plt}

\PYG{c+c1}{\PYGZsh{}\PYGZsh{}\PYGZsh{}\PYGZsh{}\PYGZsh{}\PYGZsh{}\PYGZsh{}\PYGZsh{}\PYGZsh{}\PYGZsh{}\PYGZsh{}\PYGZsh{}\PYGZsh{}\PYGZsh{}\PYGZsh{}\PYGZsh{}\PYGZsh{}\PYGZsh{}\PYGZsh{}\PYGZsh{}\PYGZsh{}\PYGZsh{}\PYGZsh{}\PYGZsh{}\PYGZsh{}\PYGZsh{}\PYGZsh{}\PYGZsh{}\PYGZsh{}\PYGZsh{}\PYGZsh{}\PYGZsh{}\PYGZsh{}}
\PYG{c+c1}{\PYGZsh{} Adjust code to fit your needs \PYGZsh{}}
\PYG{c+c1}{\PYGZsh{}\PYGZsh{}\PYGZsh{}\PYGZsh{}\PYGZsh{}\PYGZsh{}\PYGZsh{}\PYGZsh{}\PYGZsh{}\PYGZsh{}\PYGZsh{}\PYGZsh{}\PYGZsh{}\PYGZsh{}\PYGZsh{}\PYGZsh{}\PYGZsh{}\PYGZsh{}\PYGZsh{}\PYGZsh{}\PYGZsh{}\PYGZsh{}\PYGZsh{}\PYGZsh{}\PYGZsh{}\PYGZsh{}\PYGZsh{}\PYGZsh{}\PYGZsh{}\PYGZsh{}\PYGZsh{}\PYGZsh{}\PYGZsh{}}

\PYG{n}{t} \PYG{o}{=} \PYG{n}{np}\PYG{o}{.}\PYG{n}{arange}\PYG{p}{(}\PYG{l+m+mi}{0}\PYG{p}{,}\PYG{l+m+mi}{10}\PYG{p}{,}\PYG{l+m+mf}{0.1}\PYG{p}{)}
\PYG{n}{x} \PYG{o}{=} \PYG{n}{t}\PYG{o}{*}\PYG{o}{*}\PYG{l+m+mi}{2}

\PYG{n}{plt}\PYG{o}{.}\PYG{n}{plot}\PYG{p}{(}\PYG{n}{t}\PYG{p}{,}\PYG{n}{x}\PYG{p}{)}
\end{sphinxVerbatim}

\end{sphinxuseclass}\end{sphinxVerbatimInput}
\begin{sphinxVerbatimOutput}

\begin{sphinxuseclass}{cell_output}
\begin{sphinxVerbatim}[commandchars=\\\{\}]
[\PYGZlt{}matplotlib.lines.Line2D at 0x10eb99540\PYGZgt{}]
\end{sphinxVerbatim}

\noindent\sphinxincludegraphics{{activity-SHO_and_dynamical_systems_7_1}.png}

\end{sphinxuseclass}\end{sphinxVerbatimOutput}

\end{sphinxuseclass}

\subsubsection{SHO properties}
\label{\detokenize{content/2_oscillations/activity-SHO_and_dynamical_systems:sho-properties}}
\sphinxAtStartPar
You have likely work with the SHO in other contexts. You have learned about a variety of aspects of the SHO. Talk with your neighbor about different things you all know about the properties of the SHO (it’s motion, it’s energy, etc.).

\sphinxAtStartPar
\sphinxstylestrong{✅ Do this}

\sphinxAtStartPar
Decide on one or two things to investigate and plot with respect to the SHO. \sphinxstylestrong{Be ready to explain your figure, to discuss how to improve it, and what we can claim from it.}

\begin{sphinxuseclass}{cell}\begin{sphinxVerbatimInput}

\begin{sphinxuseclass}{cell_input}
\begin{sphinxVerbatim}[commandchars=\\\{\}]
\PYG{k+kn}{import} \PYG{n+nn}{numpy} \PYG{k}{as} \PYG{n+nn}{np}
\PYG{k+kn}{import} \PYG{n+nn}{matplotlib}\PYG{n+nn}{.}\PYG{n+nn}{pyplot} \PYG{k}{as} \PYG{n+nn}{plt}

\PYG{c+c1}{\PYGZsh{}\PYGZsh{}\PYGZsh{}\PYGZsh{}\PYGZsh{}\PYGZsh{}\PYGZsh{}\PYGZsh{}\PYGZsh{}\PYGZsh{}\PYGZsh{}\PYGZsh{}\PYGZsh{}\PYGZsh{}\PYGZsh{}\PYGZsh{}\PYGZsh{}\PYGZsh{}\PYGZsh{}\PYGZsh{}\PYGZsh{}\PYGZsh{}\PYGZsh{}\PYGZsh{}\PYGZsh{}\PYGZsh{}\PYGZsh{}\PYGZsh{}\PYGZsh{}\PYGZsh{}\PYGZsh{}\PYGZsh{}\PYGZsh{}}
\PYG{c+c1}{\PYGZsh{} Adjust code to fit your needs \PYGZsh{}}
\PYG{c+c1}{\PYGZsh{}\PYGZsh{}\PYGZsh{}\PYGZsh{}\PYGZsh{}\PYGZsh{}\PYGZsh{}\PYGZsh{}\PYGZsh{}\PYGZsh{}\PYGZsh{}\PYGZsh{}\PYGZsh{}\PYGZsh{}\PYGZsh{}\PYGZsh{}\PYGZsh{}\PYGZsh{}\PYGZsh{}\PYGZsh{}\PYGZsh{}\PYGZsh{}\PYGZsh{}\PYGZsh{}\PYGZsh{}\PYGZsh{}\PYGZsh{}\PYGZsh{}\PYGZsh{}\PYGZsh{}\PYGZsh{}\PYGZsh{}\PYGZsh{}}

\PYG{n}{t} \PYG{o}{=} \PYG{n}{np}\PYG{o}{.}\PYG{n}{arange}\PYG{p}{(}\PYG{l+m+mi}{0}\PYG{p}{,}\PYG{l+m+mi}{10}\PYG{p}{,}\PYG{l+m+mf}{0.1}\PYG{p}{)}
\PYG{n}{x} \PYG{o}{=} \PYG{n}{t}\PYG{o}{*}\PYG{o}{*}\PYG{l+m+mi}{2}

\PYG{n}{plt}\PYG{o}{.}\PYG{n}{plot}\PYG{p}{(}\PYG{n}{t}\PYG{p}{,}\PYG{n}{x}\PYG{p}{)}
\end{sphinxVerbatim}

\end{sphinxuseclass}\end{sphinxVerbatimInput}
\begin{sphinxVerbatimOutput}

\begin{sphinxuseclass}{cell_output}
\begin{sphinxVerbatim}[commandchars=\\\{\}]
[\PYGZlt{}matplotlib.lines.Line2D at 0x10ec95ae0\PYGZgt{}]
\end{sphinxVerbatim}

\noindent\sphinxincludegraphics{{activity-SHO_and_dynamical_systems_9_1}.png}

\end{sphinxuseclass}\end{sphinxVerbatimOutput}

\end{sphinxuseclass}

\subsubsection{Shifting gears}
\label{\detokenize{content/2_oscillations/activity-SHO_and_dynamical_systems:shifting-gears}}
\sphinxAtStartPar
We will come back to the SHO and how we complicate it soon, but we are going to use our knowledge of the SHO to bring in a new analytical tool: \sphinxhref{https://en.wikipedia.org/wiki/Dynamical\_system}{Dynamical Systems}.


\subsection{Introduction to Dynamical Systems}
\label{\detokenize{content/2_oscillations/activity-SHO_and_dynamical_systems:introduction-to-dynamical-systems}}
\sphinxAtStartPar
Dynamical systems is a branch of mathematics that seeks to investigate the time dependence of a system and characterize the families of trajectories the system can have. We don’t often solve the system for a given set of conditions in dynamical systems work. Instead, we want to \sphinxstylestrong{uncover the families of possible solutions} for a system to see what kinds of things it \sphinxstyleemphasis{can} do. An excellent book on this subject that we will draw from is Strogatz’s book on Nonlinear Dynamics {[}\hyperlink{cite.content/X_additional_pages/references-page:id6}{Strogatz, 2018}{]}; i highly recommend you find a copy for yourself.


\subsubsection{Geometric Thinking and Phase Portraits}
\label{\detokenize{content/2_oscillations/activity-SHO_and_dynamical_systems:geometric-thinking-and-phase-portraits}}
\sphinxAtStartPar
One of the first things we will use to investigate the SHO from a dynamical systems perspective is the \sphinxhref{https://en.wikipedia.org/wiki/Phase\_portrait}{phase portrait}, where we will plot the velocity against the position. We can use the phase portrait to tell us what families of solutions we might expect to see.

\sphinxAtStartPar
\sphinxstyleemphasis{Phase portraits are quite useful for second order differential equations (or any two\sphinxhyphen{}dimensional system) because we frequenty use and interpret 2D graphs.} Notice that we focused on “2nd order differential equations” not “linear 2nd order.” That is because, as we will see, \sphinxstylestrong{phase portraits are particularly useful for nonlinear 2nd order differential equations.}


\subsubsection{Example}
\label{\detokenize{content/2_oscillations/activity-SHO_and_dynamical_systems:example}}
\sphinxAtStartPar
Consider a falling object in one\sphinxhyphen{}dimension. A model for free fall where we define \(+y\) in the upward direction (away from the ground) and we neglect influences beyond the gravitational interaction leads us to:
\begin{equation*}
\begin{split}\ddot{y} = -g\end{split}
\end{equation*}
\sphinxAtStartPar
where \(g \equiv +9.8\) m/s (that is, \(g>0\) with these definitions).

\sphinxAtStartPar
We can make a phase portrait by considering how the position, \(y\), and velocity, \(v_y\), change. We represent those changes with arrows in the same space. To do this we must separate our 2nd order ODE into two 1st order ODEs (one for \(y\) and one for \(v_y\)). That is relatively straightforward for this case:
\begin{equation*}
\begin{split}\dot{y} = v_y\end{split}
\end{equation*}\begin{equation*}
\begin{split}\dot{v_y} = -g\end{split}
\end{equation*}

\paragraph{Plot the vector field}
\label{\detokenize{content/2_oscillations/activity-SHO_and_dynamical_systems:plot-the-vector-field}}
\sphinxAtStartPar
The phase portrait can be represented by arrows that indicate how the system moves through phase space. Below we have written the code to make the phase portrait for this system.

\begin{sphinxuseclass}{cell}\begin{sphinxVerbatimInput}

\begin{sphinxuseclass}{cell_input}
\begin{sphinxVerbatim}[commandchars=\\\{\}]
\PYG{k}{def} \PYG{n+nf}{GravPhasePortrait}\PYG{p}{(}\PYG{n}{Y}\PYG{p}{,}\PYG{n}{t}\PYG{p}{)}\PYG{p}{:}

\PYG{c+c1}{\PYGZsh{}\PYGZsh{} GravPhase Portrait takes the position variable y1 and velocity variable y2 }
\PYG{c+c1}{\PYGZsh{}\PYGZsh{} and returns the vectors that describe how each change in the phase space}
    
    \PYG{n}{y1}\PYG{p}{,} \PYG{n}{y2} \PYG{o}{=} \PYG{n}{Y}
    \PYG{n}{vectors} \PYG{o}{=} \PYG{p}{[}\PYG{n}{y2}\PYG{p}{,} \PYG{o}{\PYGZhy{}}\PYG{l+m+mf}{9.8}\PYG{p}{]} \PYG{c+c1}{\PYGZsh{}\PYGZsh{} Specific to the problem}
    \PYG{k}{return} \PYG{n}{vectors}

\PYG{n}{y1} \PYG{o}{=} \PYG{n}{np}\PYG{o}{.}\PYG{n}{linspace}\PYG{p}{(}\PYG{l+m+mf}{0.0}\PYG{p}{,} \PYG{l+m+mf}{10.0}\PYG{p}{,} \PYG{l+m+mi}{20}\PYG{p}{)}
\PYG{n}{y2} \PYG{o}{=} \PYG{n}{np}\PYG{o}{.}\PYG{n}{linspace}\PYG{p}{(}\PYG{o}{\PYGZhy{}}\PYG{l+m+mf}{20.0}\PYG{p}{,} \PYG{l+m+mf}{2.0}\PYG{p}{,} \PYG{l+m+mi}{20}\PYG{p}{)}

\PYG{c+c1}{\PYGZsh{}\PYGZsh{} A meshgrid makes pairs of points in a 2D space that }
\PYG{c+c1}{\PYGZsh{}\PYGZsh{} we can use to compute the value of a vector at a }
\PYG{c+c1}{\PYGZsh{}\PYGZsh{} location in 2D space}

\PYG{n}{Y1}\PYG{p}{,} \PYG{n}{Y2} \PYG{o}{=} \PYG{n}{np}\PYG{o}{.}\PYG{n}{meshgrid}\PYG{p}{(}\PYG{n}{y1}\PYG{p}{,} \PYG{n}{y2}\PYG{p}{)}

\PYG{n}{t} \PYG{o}{=} \PYG{l+m+mi}{0}

\PYG{n}{u}\PYG{p}{,} \PYG{n}{v} \PYG{o}{=} \PYG{n}{np}\PYG{o}{.}\PYG{n}{zeros}\PYG{p}{(}\PYG{n}{Y1}\PYG{o}{.}\PYG{n}{shape}\PYG{p}{)}\PYG{p}{,} \PYG{n}{np}\PYG{o}{.}\PYG{n}{zeros}\PYG{p}{(}\PYG{n}{Y2}\PYG{o}{.}\PYG{n}{shape}\PYG{p}{)}

\PYG{n}{X}\PYG{p}{,} \PYG{n}{Y} \PYG{o}{=} \PYG{n}{Y1}\PYG{o}{.}\PYG{n}{shape}

\PYG{c+c1}{\PYGZsh{}\PYGZsh{} We need to compute the vectors in the space}
\PYG{c+c1}{\PYGZsh{}\PYGZsh{} This loop steps through each grid point (not two loops)}
\PYG{c+c1}{\PYGZsh{}\PYGZsh{} to do this calculation}

\PYG{k}{for} \PYG{n}{i} \PYG{o+ow}{in} \PYG{n+nb}{range}\PYG{p}{(}\PYG{n}{X}\PYG{p}{)}\PYG{p}{:}
    \PYG{k}{for} \PYG{n}{j} \PYG{o+ow}{in} \PYG{n+nb}{range}\PYG{p}{(}\PYG{n}{Y}\PYG{p}{)}\PYG{p}{:}
        \PYG{n}{x} \PYG{o}{=} \PYG{n}{Y1}\PYG{p}{[}\PYG{n}{i}\PYG{p}{,} \PYG{n}{j}\PYG{p}{]}
        \PYG{n}{y} \PYG{o}{=} \PYG{n}{Y2}\PYG{p}{[}\PYG{n}{i}\PYG{p}{,} \PYG{n}{j}\PYG{p}{]}
        \PYG{n}{vectors} \PYG{o}{=} \PYG{n}{GravPhasePortrait}\PYG{p}{(}\PYG{p}{[}\PYG{n}{x}\PYG{p}{,} \PYG{n}{y}\PYG{p}{]}\PYG{p}{,} \PYG{n}{t}\PYG{p}{)}
        \PYG{n}{u}\PYG{p}{[}\PYG{n}{i}\PYG{p}{,}\PYG{n}{j}\PYG{p}{]} \PYG{o}{=} \PYG{n}{vectors}\PYG{p}{[}\PYG{l+m+mi}{0}\PYG{p}{]}
        \PYG{n}{v}\PYG{p}{[}\PYG{n}{i}\PYG{p}{,}\PYG{n}{j}\PYG{p}{]} \PYG{o}{=} \PYG{n}{vectors}\PYG{p}{[}\PYG{l+m+mi}{1}\PYG{p}{]}

\PYG{c+c1}{\PYGZsh{}\PYGZsh{} A quiver plot is a common way to present a vector field}
\PYG{c+c1}{\PYGZsh{}\PYGZsh{} THe arrows indicate the direction of the \PYGZdq{}flow\PYGZdq{} of the dynamical system  }
\PYG{n}{ax} \PYG{o}{=} \PYG{n}{plt}\PYG{o}{.}\PYG{n}{figure}\PYG{p}{(}\PYG{n}{figsize}\PYG{o}{=}\PYG{p}{(}\PYG{l+m+mi}{10}\PYG{p}{,}\PYG{l+m+mi}{6}\PYG{p}{)}\PYG{p}{)}
\PYG{n}{Q} \PYG{o}{=} \PYG{n}{plt}\PYG{o}{.}\PYG{n}{quiver}\PYG{p}{(}\PYG{n}{Y1}\PYG{p}{,} \PYG{n}{Y2}\PYG{p}{,} \PYG{n}{u}\PYG{p}{,} \PYG{n}{v}\PYG{p}{,} \PYG{n}{color}\PYG{o}{=}\PYG{l+s+s1}{\PYGZsq{}}\PYG{l+s+s1}{r}\PYG{l+s+s1}{\PYGZsq{}}\PYG{p}{)}

\PYG{n}{plt}\PYG{o}{.}\PYG{n}{plot}\PYG{p}{(}\PYG{p}{[}\PYG{o}{\PYGZhy{}}\PYG{l+m+mf}{.1}\PYG{p}{,}\PYG{l+m+mi}{11}\PYG{p}{]}\PYG{p}{,}\PYG{p}{[}\PYG{l+m+mi}{0}\PYG{p}{,}\PYG{l+m+mi}{0}\PYG{p}{]}\PYG{p}{,}\PYG{l+s+s1}{\PYGZsq{}}\PYG{l+s+s1}{\PYGZhy{}k}\PYG{l+s+s1}{\PYGZsq{}}\PYG{p}{)}
\PYG{n}{plt}\PYG{o}{.}\PYG{n}{plot}\PYG{p}{(}\PYG{p}{[}\PYG{l+m+mi}{0}\PYG{p}{,}\PYG{l+m+mi}{0}\PYG{p}{]}\PYG{p}{,}\PYG{p}{[}\PYG{o}{\PYGZhy{}}\PYG{l+m+mi}{21}\PYG{p}{,}\PYG{l+m+mi}{3}\PYG{p}{]}\PYG{p}{,}\PYG{l+s+s1}{\PYGZsq{}}\PYG{l+s+s1}{\PYGZhy{}k}\PYG{l+s+s1}{\PYGZsq{}}\PYG{p}{)}
\PYG{n}{plt}\PYG{o}{.}\PYG{n}{xlabel}\PYG{p}{(}\PYG{l+s+s1}{\PYGZsq{}}\PYG{l+s+s1}{\PYGZdl{}y\PYGZdl{}}\PYG{l+s+s1}{\PYGZsq{}}\PYG{p}{)}
\PYG{n}{plt}\PYG{o}{.}\PYG{n}{ylabel}\PYG{p}{(}\PYG{l+s+s1}{\PYGZsq{}}\PYG{l+s+s1}{\PYGZdl{}v\PYGZus{}y\PYGZdl{}}\PYG{l+s+s1}{\PYGZsq{}}\PYG{p}{)}

\PYG{c+c1}{\PYGZsh{}\PYGZsh{} Plotting a trajectory}
\PYG{n}{y0}\PYG{p}{,} \PYG{n}{v0}\PYG{p}{,} \PYG{n}{g} \PYG{o}{=} \PYG{l+m+mi}{10}\PYG{p}{,} \PYG{l+m+mi}{0}\PYG{p}{,} \PYG{l+m+mf}{9.8}

\PYG{c+c1}{\PYGZsh{}\PYGZsh{} To do this we solved the kinematic equations for}
\PYG{c+c1}{\PYGZsh{}\PYGZsh{} the velocity in terms of the position}
\PYG{c+c1}{\PYGZsh{}\PYGZsh{} In some cases, we can do this, in others,}
\PYG{c+c1}{\PYGZsh{}\PYGZsh{} we will have generate data based on time }
\PYG{c+c1}{\PYGZsh{}\PYGZsh{} and plot things together (see below), but a more}
\PYG{c+c1}{\PYGZsh{}\PYGZsh{} useful tool is to numerically solve the problem}

\PYG{n}{y\PYGZus{}traj} \PYG{o}{=} \PYG{n}{np}\PYG{o}{.}\PYG{n}{linspace}\PYG{p}{(}\PYG{l+m+mf}{0.1}\PYG{p}{,}\PYG{l+m+mi}{10}\PYG{p}{,}\PYG{l+m+mi}{100}\PYG{p}{)}
\PYG{n}{v\PYGZus{}traj} \PYG{o}{=} \PYG{o}{\PYGZhy{}}\PYG{l+m+mi}{1}\PYG{o}{*}\PYG{n}{np}\PYG{o}{.}\PYG{n}{sqrt}\PYG{p}{(}\PYG{n}{v0}\PYG{o}{*}\PYG{o}{*}\PYG{l+m+mi}{2}\PYG{o}{\PYGZhy{}}\PYG{l+m+mi}{2}\PYG{o}{*}\PYG{n}{g}\PYG{o}{*}\PYG{p}{(}\PYG{n}{y\PYGZus{}traj}\PYG{o}{\PYGZhy{}}\PYG{n}{y0}\PYG{p}{)}\PYG{p}{)} 
\PYG{c+c1}{\PYGZsh{}print(2*g*(y\PYGZus{}traj\PYGZhy{}y0)**2)}
\PYG{n}{plt}\PYG{o}{.}\PYG{n}{plot}\PYG{p}{(}\PYG{n}{y\PYGZus{}traj}\PYG{p}{,}\PYG{n}{v\PYGZus{}traj}\PYG{p}{,}\PYG{l+s+s1}{\PYGZsq{}}\PYG{l+s+s1}{\PYGZhy{}b}\PYG{l+s+s1}{\PYGZsq{}}\PYG{p}{)}
\end{sphinxVerbatim}

\end{sphinxuseclass}\end{sphinxVerbatimInput}
\begin{sphinxVerbatimOutput}

\begin{sphinxuseclass}{cell_output}
\begin{sphinxVerbatim}[commandchars=\\\{\}]
[\PYGZlt{}matplotlib.lines.Line2D at 0x10ed40340\PYGZgt{}]
\end{sphinxVerbatim}

\noindent\sphinxincludegraphics{{activity-SHO_and_dynamical_systems_12_1}.png}

\end{sphinxuseclass}\end{sphinxVerbatimOutput}

\end{sphinxuseclass}

\subsubsection{Discussion Question}
\label{\detokenize{content/2_oscillations/activity-SHO_and_dynamical_systems:id2}}
\sphinxAtStartPar
\sphinxstylestrong{✅ Do this}

\sphinxAtStartPar
Work with your neighbors to make sense of this code and the resulting plot. Try to write down what each line is doing and explain it to each other. Then, try to interpret the plot. Adjust the parameters and see what happens? What can you learn about a falling object from this?


\subsection{Adding Drag}
\label{\detokenize{content/2_oscillations/activity-SHO_and_dynamical_systems:adding-drag}}
\sphinxAtStartPar
We are going to explore ODEs that have multiple terms next week. But ti get started consider a 1 dimensional fall object that experiences linear drag. The mathematical model we use for this is given by the following linear 2nd order differential equation.
\begin{equation*}
\begin{split}m\ddot{x} = -m*g + \beta*\dot{x}\end{split}
\end{equation*}
\sphinxAtStartPar
Linear drag is the simplest model of drag. It is useful in only a few cases (small and slow moving objects). The \sphinxhref{https://en.wikipedia.org/wiki/Reynolds\_number\#:~:text=The\%20Reynolds\%20number\%20is\%20the,the\%20interior\%20of\%20a\%20pipe.}{REynolds number} characterizes the relationship between inertial (weight) and viscous (drag) forces. Low Reynolds? Laminar flow! High Reynolds? Turbulence! One of the most interesting papers about this is called \sphinxhref{http://www.damtp.cam.ac.uk/user/gold/pdfs/purcell.pdf}{“Life at Low Reynolds Number”} by E.M. Purcell. It’s a great, short read.

\sphinxAtStartPar
We will show (not today) that we can derive trajectories for this solution of the form:
\begin{equation*}
\begin{split}v(t) = -\dfrac{g}{\beta}+\dfrac{g}{\beta} e^{-\beta t}\end{split}
\end{equation*}\begin{equation*}
\begin{split}y(t) = \int v(t) dt = y_0 - \dfrac{g}{\beta}t + \dfrac{g}{\beta^2}\left(1-e^{-\beta t}\right)\end{split}
\end{equation*}
\sphinxAtStartPar
Below we have constructed a code that explores the phase portrait of a falling object with drag


\subsubsection{Discussion Question}
\label{\detokenize{content/2_oscillations/activity-SHO_and_dynamical_systems:id3}}
\sphinxAtStartPar
\sphinxstylestrong{✅ Do this}

\sphinxAtStartPar
Work with your neighbors to make sense of this code and the resulting plots. Try to write down what each line is doing (you’ve done this already) and explain it to each other. Then, try to interpret the plots. Adjust the parameters and see what happens? What can you learn about a falling object with drag from this?

\begin{sphinxuseclass}{cell}\begin{sphinxVerbatimInput}

\begin{sphinxuseclass}{cell_input}
\begin{sphinxVerbatim}[commandchars=\\\{\}]
\PYG{k}{def} \PYG{n+nf}{GravWithLinearDragPhasePortrait}\PYG{p}{(}\PYG{n}{Y}\PYG{p}{,}\PYG{n}{t}\PYG{p}{,}\PYG{n}{beta} \PYG{o}{=} \PYG{l+m+mf}{0.1}\PYG{p}{)}\PYG{p}{:}

\PYG{c+c1}{\PYGZsh{}\PYGZsh{} GravWithLinearDragPhasePortrait takes the position variable y1 and velocity variable y2 }
\PYG{c+c1}{\PYGZsh{}\PYGZsh{} and returns the vectors that describe how each change in the phase space}
\PYG{c+c1}{\PYGZsh{}\PYGZsh{} the parameter beta lets you adjust the strength of the linear drag}
    
    \PYG{n}{y1}\PYG{p}{,} \PYG{n}{y2} \PYG{o}{=} \PYG{n}{Y}
    \PYG{n}{vectors} \PYG{o}{=} \PYG{p}{[}\PYG{n}{y2}\PYG{p}{,} \PYG{o}{\PYGZhy{}}\PYG{l+m+mf}{9.8}\PYG{o}{\PYGZhy{}}\PYG{n}{beta}\PYG{o}{*}\PYG{n}{y2}\PYG{p}{]} \PYG{c+c1}{\PYGZsh{}\PYGZsh{} Specific to the problem}
    \PYG{k}{return} \PYG{n}{vectors}

\PYG{n}{beta} \PYG{o}{=} \PYG{l+m+mf}{0.3} \PYG{c+c1}{\PYGZsh{}\PYGZsh{} DRAG COEFF}

\PYG{c+c1}{\PYGZsh{}\PYGZsh{}\PYGZsh{}\PYGZsh{}\PYGZsh{}\PYGZsh{}\PYGZsh{}\PYGZsh{}\PYGZsh{}\PYGZsh{}\PYGZsh{}\PYGZsh{}\PYGZsh{}\PYGZsh{}\PYGZsh{}\PYGZsh{}\PYGZsh{}\PYGZsh{}\PYGZsh{}\PYGZsh{}\PYGZsh{}\PYGZsh{}\PYGZsh{}\PYGZsh{}\PYGZsh{}\PYGZsh{}\PYGZsh{}}
\PYG{c+c1}{\PYGZsh{} All the same code below \PYGZsh{}}
\PYG{c+c1}{\PYGZsh{}\PYGZsh{}\PYGZsh{}\PYGZsh{}\PYGZsh{}\PYGZsh{}\PYGZsh{}\PYGZsh{}\PYGZsh{}\PYGZsh{}\PYGZsh{}\PYGZsh{}\PYGZsh{}\PYGZsh{}\PYGZsh{}\PYGZsh{}\PYGZsh{}\PYGZsh{}\PYGZsh{}\PYGZsh{}\PYGZsh{}\PYGZsh{}\PYGZsh{}\PYGZsh{}\PYGZsh{}\PYGZsh{}\PYGZsh{}}

\PYG{n}{y1} \PYG{o}{=} \PYG{n}{np}\PYG{o}{.}\PYG{n}{linspace}\PYG{p}{(}\PYG{l+m+mf}{0.0}\PYG{p}{,} \PYG{l+m+mf}{10.0}\PYG{p}{,} \PYG{l+m+mi}{20}\PYG{p}{)}
\PYG{n}{y2} \PYG{o}{=} \PYG{n}{np}\PYG{o}{.}\PYG{n}{linspace}\PYG{p}{(}\PYG{o}{\PYGZhy{}}\PYG{l+m+mf}{20.0}\PYG{p}{,} \PYG{l+m+mf}{2.0}\PYG{p}{,} \PYG{l+m+mi}{20}\PYG{p}{)}

\PYG{n}{Y1}\PYG{p}{,} \PYG{n}{Y2} \PYG{o}{=} \PYG{n}{np}\PYG{o}{.}\PYG{n}{meshgrid}\PYG{p}{(}\PYG{n}{y1}\PYG{p}{,} \PYG{n}{y2}\PYG{p}{)}

\PYG{n}{t} \PYG{o}{=} \PYG{l+m+mi}{0}

\PYG{n}{u}\PYG{p}{,} \PYG{n}{v} \PYG{o}{=} \PYG{n}{np}\PYG{o}{.}\PYG{n}{zeros}\PYG{p}{(}\PYG{n}{Y1}\PYG{o}{.}\PYG{n}{shape}\PYG{p}{)}\PYG{p}{,} \PYG{n}{np}\PYG{o}{.}\PYG{n}{zeros}\PYG{p}{(}\PYG{n}{Y2}\PYG{o}{.}\PYG{n}{shape}\PYG{p}{)}

\PYG{n}{X}\PYG{p}{,} \PYG{n}{Y} \PYG{o}{=} \PYG{n}{Y1}\PYG{o}{.}\PYG{n}{shape}

\PYG{k}{for} \PYG{n}{i} \PYG{o+ow}{in} \PYG{n+nb}{range}\PYG{p}{(}\PYG{n}{X}\PYG{p}{)}\PYG{p}{:}
    \PYG{k}{for} \PYG{n}{j} \PYG{o+ow}{in} \PYG{n+nb}{range}\PYG{p}{(}\PYG{n}{Y}\PYG{p}{)}\PYG{p}{:}
        \PYG{n}{x} \PYG{o}{=} \PYG{n}{Y1}\PYG{p}{[}\PYG{n}{i}\PYG{p}{,} \PYG{n}{j}\PYG{p}{]}
        \PYG{n}{y} \PYG{o}{=} \PYG{n}{Y2}\PYG{p}{[}\PYG{n}{i}\PYG{p}{,} \PYG{n}{j}\PYG{p}{]}
        \PYG{n}{vectors} \PYG{o}{=} \PYG{n}{GravWithLinearDragPhasePortrait}\PYG{p}{(}\PYG{p}{[}\PYG{n}{x}\PYG{p}{,} \PYG{n}{y}\PYG{p}{]}\PYG{p}{,} \PYG{n}{t}\PYG{p}{,} \PYG{n}{beta}\PYG{p}{)}
        \PYG{n}{u}\PYG{p}{[}\PYG{n}{i}\PYG{p}{,}\PYG{n}{j}\PYG{p}{]} \PYG{o}{=} \PYG{n}{vectors}\PYG{p}{[}\PYG{l+m+mi}{0}\PYG{p}{]}
        \PYG{n}{v}\PYG{p}{[}\PYG{n}{i}\PYG{p}{,}\PYG{n}{j}\PYG{p}{]} \PYG{o}{=} \PYG{n}{vectors}\PYG{p}{[}\PYG{l+m+mi}{1}\PYG{p}{]}

\PYG{n}{ax} \PYG{o}{=} \PYG{n}{plt}\PYG{o}{.}\PYG{n}{figure}\PYG{p}{(}\PYG{n}{figsize}\PYG{o}{=}\PYG{p}{(}\PYG{l+m+mi}{10}\PYG{p}{,}\PYG{l+m+mi}{6}\PYG{p}{)}\PYG{p}{)}
\PYG{n}{Q} \PYG{o}{=} \PYG{n}{plt}\PYG{o}{.}\PYG{n}{quiver}\PYG{p}{(}\PYG{n}{Y1}\PYG{p}{,} \PYG{n}{Y2}\PYG{p}{,} \PYG{n}{u}\PYG{p}{,} \PYG{n}{v}\PYG{p}{,} \PYG{n}{color}\PYG{o}{=}\PYG{l+s+s1}{\PYGZsq{}}\PYG{l+s+s1}{k}\PYG{l+s+s1}{\PYGZsq{}}\PYG{p}{)}

\PYG{n}{plt}\PYG{o}{.}\PYG{n}{plot}\PYG{p}{(}\PYG{p}{[}\PYG{o}{\PYGZhy{}}\PYG{l+m+mf}{.1}\PYG{p}{,}\PYG{l+m+mi}{11}\PYG{p}{]}\PYG{p}{,}\PYG{p}{[}\PYG{l+m+mi}{0}\PYG{p}{,}\PYG{l+m+mi}{0}\PYG{p}{]}\PYG{p}{,}\PYG{l+s+s1}{\PYGZsq{}}\PYG{l+s+s1}{\PYGZhy{}k}\PYG{l+s+s1}{\PYGZsq{}}\PYG{p}{)}
\PYG{n}{plt}\PYG{o}{.}\PYG{n}{plot}\PYG{p}{(}\PYG{p}{[}\PYG{l+m+mi}{0}\PYG{p}{,}\PYG{l+m+mi}{0}\PYG{p}{]}\PYG{p}{,}\PYG{p}{[}\PYG{o}{\PYGZhy{}}\PYG{l+m+mi}{21}\PYG{p}{,}\PYG{l+m+mi}{3}\PYG{p}{]}\PYG{p}{,}\PYG{l+s+s1}{\PYGZsq{}}\PYG{l+s+s1}{\PYGZhy{}k}\PYG{l+s+s1}{\PYGZsq{}}\PYG{p}{)}
\PYG{n}{plt}\PYG{o}{.}\PYG{n}{xlabel}\PYG{p}{(}\PYG{l+s+s1}{\PYGZsq{}}\PYG{l+s+s1}{\PYGZdl{}y\PYGZdl{}}\PYG{l+s+s1}{\PYGZsq{}}\PYG{p}{)}
\PYG{n}{plt}\PYG{o}{.}\PYG{n}{ylabel}\PYG{p}{(}\PYG{l+s+s1}{\PYGZsq{}}\PYG{l+s+s1}{\PYGZdl{}v\PYGZus{}y\PYGZdl{}}\PYG{l+s+s1}{\PYGZsq{}}\PYG{p}{)}

\PYG{c+c1}{\PYGZsh{}\PYGZsh{}\PYGZsh{} Trajectory}
\PYG{n}{t} \PYG{o}{=} \PYG{n}{np}\PYG{o}{.}\PYG{n}{linspace}\PYG{p}{(}\PYG{l+m+mf}{0.1}\PYG{p}{,}\PYG{l+m+mi}{10}\PYG{p}{,}\PYG{l+m+mi}{1000}\PYG{p}{)}

\PYG{n}{y0}\PYG{p}{,} \PYG{n}{v0}\PYG{p}{,} \PYG{n}{g} \PYG{o}{=} \PYG{l+m+mf}{10.0}\PYG{p}{,} \PYG{l+m+mf}{0.0}\PYG{p}{,} \PYG{l+m+mf}{9.8}
\PYG{n}{v\PYGZus{}traj} \PYG{o}{=} \PYG{n}{v0}\PYG{o}{\PYGZhy{}}\PYG{n}{g}\PYG{o}{/}\PYG{n}{beta}\PYG{o}{+}\PYG{p}{(}\PYG{n}{g}\PYG{o}{/}\PYG{n}{beta}\PYG{p}{)}\PYG{o}{*}\PYG{n}{np}\PYG{o}{.}\PYG{n}{exp}\PYG{p}{(}\PYG{o}{\PYGZhy{}}\PYG{n}{beta}\PYG{o}{*}\PYG{n}{t}\PYG{p}{)}
\PYG{n}{y\PYGZus{}traj} \PYG{o}{=} \PYG{n}{y0} \PYG{o}{\PYGZhy{}} \PYG{p}{(}\PYG{n}{g}\PYG{o}{/}\PYG{n}{beta}\PYG{p}{)}\PYG{o}{*}\PYG{n}{t} \PYG{o}{+} \PYG{p}{(}\PYG{n}{g}\PYG{o}{/}\PYG{n}{beta}\PYG{o}{*}\PYG{o}{*}\PYG{l+m+mi}{2}\PYG{p}{)}\PYG{o}{*}\PYG{p}{(}\PYG{l+m+mi}{1}\PYG{o}{\PYGZhy{}}\PYG{n}{np}\PYG{o}{.}\PYG{n}{exp}\PYG{p}{(}\PYG{o}{\PYGZhy{}}\PYG{n}{beta}\PYG{o}{*}\PYG{n}{t}\PYG{p}{)}\PYG{p}{)}

\PYG{n}{y0}\PYG{p}{,} \PYG{n}{v0}\PYG{p}{,} \PYG{n}{g} \PYG{o}{=} \PYG{l+m+mf}{10.0}\PYG{p}{,} \PYG{o}{\PYGZhy{}}\PYG{l+m+mf}{2.0}\PYG{p}{,} \PYG{l+m+mf}{9.8}
\PYG{n}{v\PYGZus{}traj2} \PYG{o}{=} \PYG{n}{v0}\PYG{o}{\PYGZhy{}}\PYG{n}{g}\PYG{o}{/}\PYG{n}{beta}\PYG{o}{+}\PYG{p}{(}\PYG{n}{g}\PYG{o}{/}\PYG{n}{beta}\PYG{p}{)}\PYG{o}{*}\PYG{n}{np}\PYG{o}{.}\PYG{n}{exp}\PYG{p}{(}\PYG{o}{\PYGZhy{}}\PYG{n}{beta}\PYG{o}{*}\PYG{n}{t}\PYG{p}{)}
\PYG{n}{y\PYGZus{}traj2} \PYG{o}{=} \PYG{n}{y0} \PYG{o}{\PYGZhy{}} \PYG{p}{(}\PYG{n}{g}\PYG{o}{/}\PYG{n}{beta}\PYG{p}{)}\PYG{o}{*}\PYG{n}{t} \PYG{o}{+} \PYG{p}{(}\PYG{n}{g}\PYG{o}{/}\PYG{n}{beta}\PYG{o}{*}\PYG{o}{*}\PYG{l+m+mi}{2}\PYG{p}{)}\PYG{o}{*}\PYG{p}{(}\PYG{l+m+mi}{1}\PYG{o}{\PYGZhy{}}\PYG{n}{np}\PYG{o}{.}\PYG{n}{exp}\PYG{p}{(}\PYG{o}{\PYGZhy{}}\PYG{n}{beta}\PYG{o}{*}\PYG{n}{t}\PYG{p}{)}\PYG{p}{)}

\PYG{n}{y0}\PYG{p}{,} \PYG{n}{v0}\PYG{p}{,} \PYG{n}{g} \PYG{o}{=} \PYG{l+m+mf}{9.0}\PYG{p}{,} \PYG{o}{\PYGZhy{}}\PYG{l+m+mf}{2.0}\PYG{p}{,} \PYG{l+m+mf}{9.8}
\PYG{n}{v\PYGZus{}traj3} \PYG{o}{=} \PYG{n}{v0}\PYG{o}{\PYGZhy{}}\PYG{n}{g}\PYG{o}{/}\PYG{n}{beta}\PYG{o}{+}\PYG{p}{(}\PYG{n}{g}\PYG{o}{/}\PYG{n}{beta}\PYG{p}{)}\PYG{o}{*}\PYG{n}{np}\PYG{o}{.}\PYG{n}{exp}\PYG{p}{(}\PYG{o}{\PYGZhy{}}\PYG{n}{beta}\PYG{o}{*}\PYG{n}{t}\PYG{p}{)}
\PYG{n}{y\PYGZus{}traj3} \PYG{o}{=} \PYG{n}{y0} \PYG{o}{\PYGZhy{}} \PYG{p}{(}\PYG{n}{g}\PYG{o}{/}\PYG{n}{beta}\PYG{p}{)}\PYG{o}{*}\PYG{n}{t} \PYG{o}{+} \PYG{p}{(}\PYG{n}{g}\PYG{o}{/}\PYG{n}{beta}\PYG{o}{*}\PYG{o}{*}\PYG{l+m+mi}{2}\PYG{p}{)}\PYG{o}{*}\PYG{p}{(}\PYG{l+m+mi}{1}\PYG{o}{\PYGZhy{}}\PYG{n}{np}\PYG{o}{.}\PYG{n}{exp}\PYG{p}{(}\PYG{o}{\PYGZhy{}}\PYG{n}{beta}\PYG{o}{*}\PYG{n}{t}\PYG{p}{)}\PYG{p}{)}


\PYG{n}{plt}\PYG{o}{.}\PYG{n}{plot}\PYG{p}{(}\PYG{n}{y\PYGZus{}traj}\PYG{p}{,} \PYG{n}{v\PYGZus{}traj}\PYG{p}{,}\PYG{l+s+s1}{\PYGZsq{}}\PYG{l+s+s1}{\PYGZhy{}b}\PYG{l+s+s1}{\PYGZsq{}}\PYG{p}{)}
\PYG{n}{plt}\PYG{o}{.}\PYG{n}{plot}\PYG{p}{(}\PYG{n}{y\PYGZus{}traj2}\PYG{p}{,} \PYG{n}{v\PYGZus{}traj2}\PYG{p}{,}\PYG{l+s+s1}{\PYGZsq{}}\PYG{l+s+s1}{\PYGZhy{}g}\PYG{l+s+s1}{\PYGZsq{}}\PYG{p}{)}
\PYG{n}{plt}\PYG{o}{.}\PYG{n}{plot}\PYG{p}{(}\PYG{n}{y\PYGZus{}traj3}\PYG{p}{,} \PYG{n}{v\PYGZus{}traj3}\PYG{p}{,}\PYG{l+s+s1}{\PYGZsq{}}\PYG{l+s+s1}{\PYGZhy{}r}\PYG{l+s+s1}{\PYGZsq{}}\PYG{p}{)}


\PYG{n}{plt}\PYG{o}{.}\PYG{n}{xlabel}\PYG{p}{(}\PYG{l+s+s1}{\PYGZsq{}}\PYG{l+s+s1}{\PYGZdl{}y\PYGZdl{}}\PYG{l+s+s1}{\PYGZsq{}}\PYG{p}{)}
\PYG{n}{plt}\PYG{o}{.}\PYG{n}{ylabel}\PYG{p}{(}\PYG{l+s+s1}{\PYGZsq{}}\PYG{l+s+s1}{\PYGZdl{}v\PYGZus{}y\PYGZdl{}}\PYG{l+s+s1}{\PYGZsq{}}\PYG{p}{)}

\PYG{n}{plt}\PYG{o}{.}\PYG{n}{xlim}\PYG{p}{(}\PYG{p}{[}\PYG{o}{\PYGZhy{}}\PYG{l+m+mi}{1}\PYG{p}{,} \PYG{l+m+mi}{11}\PYG{p}{]}\PYG{p}{)}
\PYG{n}{plt}\PYG{o}{.}\PYG{n}{ylim}\PYG{p}{(}\PYG{p}{[}\PYG{o}{\PYGZhy{}}\PYG{l+m+mi}{20}\PYG{p}{,} \PYG{l+m+mi}{5}\PYG{p}{]}\PYG{p}{)}
\end{sphinxVerbatim}

\end{sphinxuseclass}\end{sphinxVerbatimInput}
\begin{sphinxVerbatimOutput}

\begin{sphinxuseclass}{cell_output}
\begin{sphinxVerbatim}[commandchars=\\\{\}]
(\PYGZhy{}20.0, 5.0)
\end{sphinxVerbatim}

\noindent\sphinxincludegraphics{{activity-SHO_and_dynamical_systems_15_1}.png}

\end{sphinxuseclass}\end{sphinxVerbatimOutput}

\end{sphinxuseclass}

\subsubsection{Discussion Question}
\label{\detokenize{content/2_oscillations/activity-SHO_and_dynamical_systems:id4}}
\sphinxAtStartPar
\sphinxstylestrong{✅ Do this}

\sphinxAtStartPar
Compare the phase potraits of the no drag and drag case. What do they tell you about the families of solutions? Discuss with your neighbors.


\subsection{Back to the SHO}
\label{\detokenize{content/2_oscillations/activity-SHO_and_dynamical_systems:back-to-the-sho}}
\sphinxAtStartPar
\sphinxstylestrong{✅ Do this}

\sphinxAtStartPar
Use the above code to make a phase portrait of the SHO. Add trajectories to the portrait. What does this tell you about the SHO?

\begin{sphinxuseclass}{cell}\begin{sphinxVerbatimInput}

\begin{sphinxuseclass}{cell_input}
\begin{sphinxVerbatim}[commandchars=\\\{\}]
\PYG{c+c1}{\PYGZsh{}\PYGZsh{} Your code here}
\end{sphinxVerbatim}

\end{sphinxuseclass}\end{sphinxVerbatimInput}

\end{sphinxuseclass}

\subsection{Lingering Questions}
\label{\detokenize{content/2_oscillations/activity-SHO_and_dynamical_systems:lingering-questions}}
\sphinxAtStartPar
\sphinxstylestrong{✅ Do this}

\sphinxAtStartPar
What lingering questions do you have? What should we explore more?

\sphinxAtStartPar
Notes here

\sphinxstepscope


\chapter{Works Cited}
\label{\detokenize{content/X_additional_pages/references-page:works-cited}}\label{\detokenize{content/X_additional_pages/references-page::doc}}
\begin{sphinxthebibliography}{Tra09}
\bibitem[Tra09]{content/X_additional_pages/references-page:id4}
\sphinxAtStartPar
Sharon Traweek. \sphinxstyleemphasis{Beamtimes and Lifetimes}. Harvard University Press, 2009.
\bibitem[Car83]{content/X_additional_pages/references-page:id5}
\sphinxAtStartPar
Nancy Cartwright. \sphinxstyleemphasis{How the Laws of Physics Lie}. OUP Oxford, 1983.
\bibitem[Nah11]{content/X_additional_pages/references-page:id2}
\sphinxAtStartPar
Paul J Nahin. \sphinxstyleemphasis{Dr. Euler's Fabulous Formula}. Princeton University Press, 2011.
\bibitem[Sti17]{content/X_additional_pages/references-page:id3}
\sphinxAtStartPar
David Stipp. \sphinxstyleemphasis{A most elegant equation: Euler's formula and the beauty of mathematics}. Hachette UK, 2017.
\bibitem[Str18]{content/X_additional_pages/references-page:id6}
\sphinxAtStartPar
Steven H Strogatz. \sphinxstyleemphasis{Nonlinear dynamics and chaos: with applications to physics, biology, chemistry, and engineering}. CRC press, 2018.
\end{sphinxthebibliography}







\renewcommand{\indexname}{Index}
\printindex
\end{document}