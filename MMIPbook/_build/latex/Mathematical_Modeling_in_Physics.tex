%% Generated by Sphinx.
\def\sphinxdocclass{jupyterBook}
\documentclass[letterpaper,10pt,english]{jupyterBook}
\ifdefined\pdfpxdimen
   \let\sphinxpxdimen\pdfpxdimen\else\newdimen\sphinxpxdimen
\fi \sphinxpxdimen=.75bp\relax
\ifdefined\pdfimageresolution
    \pdfimageresolution= \numexpr \dimexpr1in\relax/\sphinxpxdimen\relax
\fi
%% let collapsible pdf bookmarks panel have high depth per default
\PassOptionsToPackage{bookmarksdepth=5}{hyperref}
%% turn off hyperref patch of \index as sphinx.xdy xindy module takes care of
%% suitable \hyperpage mark-up, working around hyperref-xindy incompatibility
\PassOptionsToPackage{hyperindex=false}{hyperref}
%% memoir class requires extra handling
\makeatletter\@ifclassloaded{memoir}
{\ifdefined\memhyperindexfalse\memhyperindexfalse\fi}{}\makeatother

\PassOptionsToPackage{warn}{textcomp}

\catcode`^^^^00a0\active\protected\def^^^^00a0{\leavevmode\nobreak\ }
\usepackage{cmap}
\usepackage{fontspec}
\defaultfontfeatures[\rmfamily,\sffamily,\ttfamily]{}
\usepackage{amsmath,amssymb,amstext}
\usepackage{polyglossia}
\setmainlanguage{english}



\setmainfont{FreeSerif}[
  Extension      = .otf,
  UprightFont    = *,
  ItalicFont     = *Italic,
  BoldFont       = *Bold,
  BoldItalicFont = *BoldItalic
]
\setsansfont{FreeSans}[
  Extension      = .otf,
  UprightFont    = *,
  ItalicFont     = *Oblique,
  BoldFont       = *Bold,
  BoldItalicFont = *BoldOblique,
]
\setmonofont{FreeMono}[
  Extension      = .otf,
  UprightFont    = *,
  ItalicFont     = *Oblique,
  BoldFont       = *Bold,
  BoldItalicFont = *BoldOblique,
]



\usepackage[Bjarne]{fncychap}
\usepackage[,numfigreset=1,mathnumfig]{sphinx}

\fvset{fontsize=\small}
\usepackage{geometry}


% Include hyperref last.
\usepackage{hyperref}
% Fix anchor placement for figures with captions.
\usepackage{hypcap}% it must be loaded after hyperref.
% Set up styles of URL: it should be placed after hyperref.
\urlstyle{same}


\usepackage{sphinxmessages}



        % Start of preamble defined in sphinx-jupyterbook-latex %
         \usepackage[Latin,Greek]{ucharclasses}
        \usepackage{unicode-math}
        % fixing title of the toc
        \addto\captionsenglish{\renewcommand{\contentsname}{Contents}}
        \hypersetup{
            pdfencoding=auto,
            psdextra
        }
        % End of preamble defined in sphinx-jupyterbook-latex %
        

\title{Mathematical Modeling in Physics}
\date{Mar 07, 2022}
\release{}
\author{Danny Caballero}
\newcommand{\sphinxlogo}{\vbox{}}
\renewcommand{\releasename}{}
\makeindex
\begin{document}

\pagestyle{empty}
\sphinxmaketitle
\pagestyle{plain}
\sphinxtableofcontents
\pagestyle{normal}
\phantomsection\label{\detokenize{content/intro::doc}}






\sphinxAtStartPar
In this course, you will learn to:
\begin{itemize}
\item {} 
\sphinxAtStartPar
Develop models and stuff

\item {} 
\sphinxAtStartPar
Collaborate

\end{itemize}

\sphinxAtStartPar
The rest of this JupyterBook is (currently) organized as follows:
\begin{itemize}
\item {} 
\sphinxAtStartPar
{\hyperref[\detokenize{content/0_course/syllabus::doc}]{\sphinxcrossref{Overview of PHY 415}}}

\item {} 
\sphinxAtStartPar
{\hyperref[\detokenize{content/1_modeling/what_is_modeling::doc}]{\sphinxcrossref{What is Mathematical Modeling?}}}

\end{itemize}


\chapter{Overview of PHY 415}
\label{\detokenize{content/0_course/syllabus:overview-of-phy-415}}\label{\detokenize{content/0_course/syllabus::doc}}
\sphinxAtStartPar
Histroically, PHY 415 has focused on developing specialized analytical solutions to common physics problems and situations. However, the work of phyiscists involves a variety of tools the \sphinxstylestrong{model the behavior of physical systems}. Hence, this course will emphasize how we go about building mathematical models of physical phenomenon and how we can investigate them through a variety of lenses: analytical mathematics, graphics and visualization, as well as computational modeling using Python. By emphasizing the critical aspects of the modeling process, this course aims to develop your expertise in the process of doing physics rather than specific tools or techniques. That being said, you will also learn many differenty analytical and computaitonal tools to use for your own aims.


\section{What you should expect}
\label{\detokenize{content/0_course/syllabus:what-you-should-expect}}
\sphinxAtStartPar
In redesigning this course, I plan to emphasize more independent learning on your part and greater agency for you in determining what you learn and how you demonstrate you have learned. So you should expect:
\begin{itemize}
\item {} 
\sphinxAtStartPar
to read a variety of courses to coordinate information

\item {} 
\sphinxAtStartPar
to present your ideas publicly and to discuss them

\item {} 
\sphinxAtStartPar
to learn new approaches and novel techniques on your own

\item {} 
\sphinxAtStartPar
to become more expert than me in the areas of your interest

\item {} 
\sphinxAtStartPar
to learn more about scientists that you have not learned about

\end{itemize}

\sphinxAtStartPar
This is not to say that you are on your own. Here’s what you can expect from me:
\begin{itemize}
\item {} 
\sphinxAtStartPar
resources, information, and tools to help you learn

\item {} 
\sphinxAtStartPar
support and scaffolding to move you towards more independence in your learning

\item {} 
\sphinxAtStartPar
timely and detailed feedback to help you along

\item {} 
\sphinxAtStartPar
a commitment to an inclusive classroom

\end{itemize}


\section{Contact Information}
\label{\detokenize{content/0_course/syllabus:contact-information}}\begin{itemize}
\item {} 
\sphinxAtStartPar
Instructor: \sphinxhref{http://dannycab.github.io}{Prof. Danny Caballero} (he/him/his)

\item {} 
\sphinxAtStartPar
Class Meetings: Tuesdays and Thursdays 10:20am\sphinxhyphen{}12:10pm (Location: TBD)

\item {} 
\sphinxAtStartPar
Email: \sphinxhref{mailto:caball14@msu.edu}{caball14@msu.edu}, cell: (517) 420\sphinxhyphen{}5330, office:
1310\sphinxhyphen{}A BPS

\item {} 
\sphinxAtStartPar
Office hrs: Open door policy. I enjoy visiting and talking with you about physics.

\item {} 
\sphinxAtStartPar
Web page:
\sphinxhref{http://dannycab.github.io/phy415msu/}{dannycab.github.io/phy415msu/}

\end{itemize}


\subsection{Slack Team}
\label{\detokenize{content/0_course/syllabus:slack-team}}
\sphinxAtStartPar
\sphinxstyleemphasis{This term we will be using Slack for class discussion.} The system is highly catered to getting you help fast and efficiently from classmates and myself. Rather than emailing questions, I encourage you to post your questions on there. MORE SOON.


\section{Grading}
\label{\detokenize{content/0_course/syllabus:grading}}
\sphinxAtStartPar
Details about {\hyperref[\detokenize{content/0_course/design::doc}]{\sphinxcrossref{\DUrole{doc,std,std-doc}{course activities are here}}}} and {\hyperref[\detokenize{content/0_course/assessments::doc}]{\sphinxcrossref{\DUrole{doc,std,std-doc}{information regarding asessment is here}}}}. How your grade is calculated appears below.

\sphinxAtStartPar
\sphinxstylestrong{NOT READY YET}


\begin{savenotes}\sphinxattablestart
\centering
\begin{tabulary}{\linewidth}[t]{|T|T|T|}
\hline
\sphinxstyletheadfamily 
\sphinxAtStartPar
Activity
&\sphinxstyletheadfamily 
\sphinxAtStartPar
Date
&\sphinxstyletheadfamily 
\sphinxAtStartPar
Percent of Grade
\\
\hline
\sphinxAtStartPar
Homework (14 total; 1 dropped)
&
\sphinxAtStartPar
Due Fri at start of class
&
\sphinxAtStartPar
40\%
\\
\hline
\sphinxAtStartPar
In\sphinxhyphen{}Class Quizzes  (7 total; 1 dropped)
&
\sphinxAtStartPar
Every other Friday
&
\sphinxAtStartPar
20\%
\\
\hline
\sphinxAtStartPar
Individual Project
&
\sphinxAtStartPar
Fri Feb 28
&
\sphinxAtStartPar
20\%
\\
\hline
\sphinxAtStartPar
Pair Project
&
\sphinxAtStartPar
Tues Apr 28
&
\sphinxAtStartPar
20\%
\\
\hline
\end{tabulary}
\par
\sphinxattableend\end{savenotes}


\section{Course Objectives}
\label{\detokenize{content/0_course/goals:course-objectives}}\label{\detokenize{content/0_course/goals::doc}}
\sphinxAtStartPar
This course emphasizes making models of physical phenomenon and how we use various tools at our disposal to investigate those models. Hence, we have learning objectives for making models of these systems and for learning specific tools.

\begin{sphinxadmonition}{note}{Principle Learning Objectives}

\sphinxAtStartPar
Students will demonstrate they can:
\begin{itemize}
\item {} 
\sphinxAtStartPar
investigate physical systems of their choosing using a variety of tools and approaches

\item {} 
\sphinxAtStartPar
construct and document a reproducible process for those investigations

\item {} 
\sphinxAtStartPar
use analytical, computational, and graphical approaches to answer specific questions in those investigations

\item {} 
\sphinxAtStartPar
provide evidence of the quality of their work using a variety of sources

\item {} 
\sphinxAtStartPar
collaborate effectively and contribute to a inclusive learning environment

\end{itemize}
\end{sphinxadmonition}

\sphinxAtStartPar
Each of these learning objectives contributes to your development as a physicist. I recognize that these are \sphinxstylestrong{big} ideas to think about. What I mean is that the objectives above are quite broad and you might be able to see a little about what or why they are included. But, below, I added more detail about each one along with a smaller scale list of objectives that you will engage with. Throughout our course, you will have opprtunties to demonstrate these objectives in your work. My aim is to make what you are assessed on in this course something you are interested in, so these objectives reflect that.


\subsection{Investigate physical systems}
\label{\detokenize{content/0_course/goals:investigate-physical-systems}}
\sphinxAtStartPar
Clearly, the principal goal we are concerned with in this course learning how to make models of physical systems. This means learning about and developing fluency with a wide variety of mathematical and computational tools. In this courses, we will make extensive use of \sphinxhref{http://anaconda.org}{Jupyter notebooks} for homework and projects. In fact, what you are reading is a set of Jupyter notebooks! Below, you will see the list of objectives for this principal objective.

\sphinxAtStartPar
Students will demonstrate they can:
\begin{itemize}
\item {} 
\sphinxAtStartPar
use mathematical techniques to predict or explain some physical phenomenon
\begin{itemize}
\item {} 
\sphinxAtStartPar
solve ordinary differential equations

\end{itemize}

\end{itemize}


\subsection{Construct and document a reproducible process}
\label{\detokenize{content/0_course/goals:construct-and-document-a-reproducible-process}}

\subsection{Use analytical, computational, and graphical approaches}
\label{\detokenize{content/0_course/goals:use-analytical-computational-and-graphical-approaches}}

\subsection{Provide evidence of the quality of their work}
\label{\detokenize{content/0_course/goals:provide-evidence-of-the-quality-of-their-work}}

\subsection{Collaborate effectively}
\label{\detokenize{content/0_course/goals:collaborate-effectively}}

\section{Course Design}
\label{\detokenize{content/0_course/design:course-design}}\label{\detokenize{content/0_course/design::doc}}
\sphinxAtStartPar
For most of you, 482 is an elective course that you are taking to advance your knowledge of electromagnetism. As such, this course is designed under several different principles than a standard course. Below, I provide those principles and their rationale.
\begin{itemize}
\item {} 
\sphinxAtStartPar
482 should help you learn the central tenets of Electromagnetism
\begin{itemize}
\item {} 
\sphinxAtStartPar
This course provides the conclusion of the story started in 481. In 481, we were concerned with static charges and steady currents, but in 482, we broaden our scope to allow distributions and thus fields to change with time. This course should help you see how all the different aspects of Electromagnetism are connected together and how it is that Classical Electromagnetism is a coherent theory. Because Electromagnetism is such a broad topic, we will cover the basic in class and you will research advanced topics and share them with your class mates. At the end of the semester, we will bring it all together.

\end{itemize}

\item {} 
\sphinxAtStartPar
482 should be a celebration of your knowledge
\begin{itemize}
\item {} 
\sphinxAtStartPar
For most of you, this course concludes your study of physics at MSU. What you have achieved in the last four years should be celebrated and enjoyed. This course will provide ample opportunities for you to share what things you know and what things you are learning with me and with each other.

\end{itemize}

\item {} 
\sphinxAtStartPar
482 should give you opportunities to engage in professional practice
\begin{itemize}
\item {} 
\sphinxAtStartPar
As you start towards your professional career, it’s important to learn what professional scientists do. You have probably already begun this work in advanced lab and research projects that you have worked on. We will continue developing your professional skills in this course through the use of course projects in lieu of exams.

\end{itemize}

\item {} 
\sphinxAtStartPar
482 will illustrate that we can learn from each other
\begin{itemize}
\item {} 
\sphinxAtStartPar
Even though I’ve been learning physics for almost 20 years, I don’t know everything. I am excited to learn from you and I hope that you are excited to learn from me and each other.

\end{itemize}

\end{itemize}


\subsection{Required purchases:}
\label{\detokenize{content/0_course/design:required-purchases}}\begin{enumerate}
\sphinxsetlistlabels{\arabic}{enumi}{enumii}{}{.}%
\item {} 
\sphinxAtStartPar
J.D. Griffiths. \sphinxhref{http://goo.gl/iU6MdA}{\sphinxstyleemphasis{Introduction to Electromagnetism}, 4th
Edition} (Pearson; 2012). This book is
pedagogically excellent and is my favorite undergrad textbook of
them all! There are \sphinxhref{http://goo.gl/78y9jw}{other editions that might be less
expensive}, and they can be substituted. But
reading assignments will come from the 4th edition, which may have
some different content.

\item {} 
\sphinxAtStartPar
An “iClicker”, available at the bookstore, will be used
every lecture.

\end{enumerate}


\section{Course Activities}
\label{\detokenize{content/0_course/design:course-activities}}

\subsection{Readings}
\label{\detokenize{content/0_course/design:readings}}
\sphinxAtStartPar
Reading is an essential part of 482! Reading the text before class is
very important. Lecture is to clarify your understanding, to help you
make sense of the material. I will assume you have done the required
readings in advance! Griffiths is one of the best (and most readable)
texts I know of \sphinxhyphen{} it will make a huge difference if you spend the time
and effort to carefully read and follow the text. The
\DUrole{xref,myst}{calendar} has the details on reading assignments.


\subsection{Class Meetings}
\label{\detokenize{content/0_course/design:class-meetings}}
\sphinxAtStartPar
\sphinxstylestrong{Classroom Etiquette:} Please silence your cell phones when entering the classroom.
I don’t mind you using them, but they can very distracting to your neighbors, so
use your judgement. I appreciate that you might have questions or comments about things in
class. If you and your neighbors are confused, just raise your hand and ask. If you are confused, you are likely not the only one and it’s better to chat about it, then move on.
Questions in lecture are always good, and are strongly encouraged!

\sphinxAtStartPar
\sphinxstylestrong{Clickers:} Occasionally, clicker questions will be used in class to gauge your understanding of a topic or concept. I do not penalize you for not knowing the correct answer. I prefer to know what you know in\sphinxhyphen{}the\sphinxhyphen{}monent. Thus, clickers will be ungraded.

\sphinxAtStartPar
\sphinxstylestrong{In\sphinxhyphen{}Class Activities:} We will also use a variety of in\sphinxhyphen{}class activities and worksheets that help you construct an understanding of a particular topic or concept. These will not be collected or graded, but we will discuss the solutions in class.


\subsection{Homework}
\label{\detokenize{content/0_course/design:homework}}
\sphinxAtStartPar
There will be a homework due every Friday by 5pm. Late homework can’t be accepted once solutions are posted \sphinxhyphen{} but, your lowest score will be dropped. Homework is exceedingly important for developing an understanding of the course material, not to mention building skills in complex physical and mathematical problem solving. They will require considerable time and personal effort this term! \sphinxstyleemphasis{Your lowest homework grade will be dropped.}

\sphinxAtStartPar
There are four kinds of homework problems in this class:

\sphinxAtStartPar
\sphinxstylestrong{Standard Homework Problems}: These are regular back\sphinxhyphen{}of\sphinxhyphen{}the\sphinxhyphen{}book type homework problems that involve derivations, calculations, figures, and graphs. If you took 481, there will be fewer of these in 482. Each question will be coarsely graded for “completion”:
\begin{itemize}
\item {} 
\sphinxAtStartPar
10 pts. complete

\item {} 
\sphinxAtStartPar
8 pts. right idea, but incomplete

\item {} 
\sphinxAtStartPar
4 pts. relatively incomplete

\item {} 
\sphinxAtStartPar
0 pts. not turned in

\end{itemize}

\sphinxAtStartPar
\sphinxstylestrong{Computational Homework Problems:} There will be \sphinxstyleemphasis{some use of computation in this course} on homework problems. I will encourage and support the use of Python (through \sphinxhref{http://jupyter.org/}{Jupyter notebooks}). You do not need any computational experience for this course as you will learn some fundamentals early on and keep using them throughout the course. You are welcome to use any environment of your choosing (e.g., Octave/MATLAB, Mathematica, C++), but I will only provide support for Python. Python is in use across the sciences, but it is becoming much used in physics, so learning it will serve you well in your future work. I suggest downloading the \sphinxhref{https://www.continuum.io/}{Anaconda distribution of Python} as it comes with all the packages you will need to get up and running with Jupyter notebooks. These will be graded on the same 10\sphinxhyphen{}8\sphinxhyphen{}4\sphinxhyphen{}0 scale as standard homework problems. Here are \sphinxhref{http://jupyter.readthedocs.io/en/latest/install.html}{instructions for installing Jupyter Notebooks}.

\sphinxAtStartPar
\sphinxstylestrong{Project Homework Problems}: These are homework problems to support your working towards completing your individual and paired projects (see below). \sphinxstyleemphasis{Projects are difficult to complete, so having some regular check\sphinxhyphen{}ins on how those projects are going, setting milestones to complete, and producing a semi\sphinxhyphen{}complete piece of a project are all important aspects of research!} These homework problems are meant to help you make that progress each week. They count twice as much as normal homework problems, but follow a similar grading scale:
\begin{itemize}
\item {} 
\sphinxAtStartPar
20 pts. complete

\item {} 
\sphinxAtStartPar
16 pts. right idea, but incomplete

\item {} 
\sphinxAtStartPar
8 pts. relatively incomplete

\item {} 
\sphinxAtStartPar
0 pts. not turned in

\end{itemize}

\sphinxAtStartPar
You will be given detailed feedback on these project homework problems as you are working on bigger projects throughout this class. You should read and be responsive to this feedback as it will help you develop a strong project.

\sphinxAtStartPar
\sphinxstylestrong{Self\sphinxhyphen{}reflection Homework Problems}: These are homework problems that ask you to evaluate your progress on your projects and how you and your partner are working together. \sphinxstyleemphasis{Evaluating how well you understand something and what you need to do to move forward is a hard thing to learn. So is working on a team (or with a partner).} These homework problems are meant to help you do both and get feedback from me on how things are going. These problems are graded out of 10 points like regular homework problems on the same 10\sphinxhyphen{}8\sphinxhyphen{}4\sphinxhyphen{}0 scale. You will also be given detailed feedback on these homework problems.

\sphinxAtStartPar
\sphinxstyleemphasis{I strongly encourage collaboration}, an essential skill in science and engineering (and highly valued by employers!) Social interactions are critical to scientists’ success – most good ideas grow out of discussions with colleagues, and essentially all physicists work as part of a group. Find partners and work on homework together. However, it is also important that you OWN the material. I strongly suggest you start homework by yourself (and that means really making an extended effort on every problem). Then work with a group, and finally, finish up on your own – write up your own work, in your own way. There will also be time for peer discussion during classes – as you work together, try to help your partners get over confusions, listen to them, ask each other questions, critique, teach each other. You will learn a lot this way! For all assignments, the work you turn in must in the end be your own: in your own words, reflecting your own understanding. (If, at any time, for any reason, you feel disadvantaged or isolated, contact me and I can discretely try to help arrange study groups.)


\subsubsection{Help Session}
\label{\detokenize{content/0_course/design:help-session}}
\sphinxAtStartPar
Help sessions/office hours are to facilitate your learning. We encourage attendance \sphinxhyphen{} plan on working in small groups, our role will be as learning coaches. The sessions are homework\sphinxhyphen{}centric, but we will not be explicitly telling anyone how to do the homework (how would that help you learn?) I strongly encourage you to start all problems on your own. If you come to help sessions “cold”, the value of homework to you will be greatly reduced.


\section{Assessments}
\label{\detokenize{content/0_course/assessments:assessments}}\label{\detokenize{content/0_course/assessments::doc}}

\subsection{In\sphinxhyphen{}Class Quizzes}
\label{\detokenize{content/0_course/assessments:in-class-quizzes}}
\sphinxAtStartPar
Every other Friday, we will have a short, in\sphinxhyphen{}class quiz that covers the material discussed in the previous two weeks. The quiz will take the form of a typical exam\sphinxhyphen{}style question – more straight\sphinxhyphen{}forward than your homework questions with not much substantive calculations. I will inform you of the type and topic of the in\sphinxhyphen{}class quiz on the Friday prior to the quiz, so you will have a week to prepare should you want. There will seven of these quizzes. \sphinxstyleemphasis{Your lowest quiz grade will be dropped.}


\subsection{Projects}
\label{\detokenize{content/0_course/assessments:projects}}
\sphinxAtStartPar
In lieu of examinations, which are not at all representative of professional physics practice, you will produce two projects.


\subsubsection{Individual Project}
\label{\detokenize{content/0_course/assessments:individual-project}}
\sphinxAtStartPar
The first project is an individual research project and is meant to mimic the kind of literature review that is needed to understand a topic that is new to you. In a nutshell, you will select a topic of active research in electromagnetism, read several journal articles pertaining to the topic, and write a 4\sphinxhyphen{}5 page summary with references about the topic. You will be working to answer the following questions in your paper:
\begin{itemize}
\item {} 
\sphinxAtStartPar
What is the phenomenon and why is it interesting?

\item {} 
\sphinxAtStartPar
What the relevant electrodynamic models that are used to make sense of this phenomenon?

\item {} 
\sphinxAtStartPar
What are the assumptions that go into these models?

\item {} 
\sphinxAtStartPar
What mathematical models and mathematical tools are used to make predictions?

\item {} 
\sphinxAtStartPar
What are some of the major predictions?

\item {} 
\sphinxAtStartPar
What experimental work has gone into validating these predictions?

\item {} 
\sphinxAtStartPar
What the challenges in connecting the experimental and theoretical predictions?

\end{itemize}

\sphinxAtStartPar
There will be seven homework questions that help you develop your individual project. \sphinxstyleemphasis{Having deadlines and milestones for such a project is important, so that you don’t get behind.} Here’s a \sphinxstyleemphasis{preliminary} listing of the homework questions that will appear (note these are subject to change!):
\begin{itemize}
\item {} 
\sphinxAtStartPar
\sphinxstylestrong{Homework 1} (What is interesting?) \sphinxhyphen{} Define your phenomenon; what is it? Why is it interesting to you? What are a few papers that you can use to start your background research (give actual references). 2\sphinxhyphen{}3 paragraphs along with a listing of at least 4 relevant journal articles.

\item {} 
\sphinxAtStartPar
\sphinxstylestrong{Homework 2} (Towards an annotated bibliography I) \sphinxhyphen{} Summarize 2 of the 4 relevant journal articles. What do they say about your phenomenon? How are the theoretical models constructed? What are the predictions and implications? 2\sphinxhyphen{}3 paragraphs per article.

\item {} 
\sphinxAtStartPar
\sphinxstylestrong{Homework 3} (Towards an annotated bibliography II)\sphinxhyphen{} Summarize 2 more relevant journal articles (can be the remaining 2 or 2 new ones if the direction of your research has changed). What do they say about your phenomenon? How are the theoretical models constructed? What are the predictions and implications? 2\sphinxhyphen{}3 paragraphs per article.

\item {} 
\sphinxAtStartPar
\sphinxstylestrong{Homework 4} (Building your paper I) \sphinxhyphen{} Summarize your background research so far (3\sphinxhyphen{}4 paragraphs on what you have learned so far) with references. What are people saying about this phenomenon? What are the relevant models? How are the models described?

\item {} 
\sphinxAtStartPar
\sphinxstylestrong{Homework 5} (Building your paper II) \sphinxhyphen{} Summarize the models used to describe your phenomenon (3\sphinxhyphen{}4 more paragraphs on what you have learned so far) with references. What are the predictions of these models? What mathematical tools are used to make these predictions?

\item {} 
\sphinxAtStartPar
\sphinxstylestrong{Homework 6} (Building your paper III) \sphinxhyphen{} Summarize the experimental work that validates the theoretical predictions (3\sphinxhyphen{}4 more paragraphs on what you have learned so far). What are the difficulties or successes connecting theory and experiment?

\item {} 
\sphinxAtStartPar
\sphinxstylestrong{Homework 7} (Constructing an abstract) \sphinxhyphen{} Summarize your entire paper in a single paragraph \sphinxhyphen{} You are writing an abstract. It should be self\sphinxhyphen{}contained and describe the entire paper in just a few sentences.

\end{itemize}

\sphinxAtStartPar
After Homework 7 (Fri. Feb. 28), your paper is due. It should be 4\sphinxhyphen{}5 pages long not counting figures, equations, and references! It should fully describe all the aspects of the phenomenon that are being modeled including how the phenomenon is modeled, what predictions/implications they are, how it is connected to experiments, and what limitations there are in the modeling of it. There should be about 6\sphinxhyphen{}10 references to articles appearing in your paper. You have been continually reading about the topic, right?


\subsubsection{Grading the individual project}
\label{\detokenize{content/0_course/assessments:grading-the-individual-project}}
\sphinxAtStartPar
A rubric for the individual project \DUrole{xref,myst}{appears here}. Notice that the rubric emphasizes several aspects of the paper with different weights. Each category will be scored on the following scale (4.0, 3.5, 3.0, 2.0, 0.0) and then averaged together using the weights for each category. Your final grade will be this averaged score converted to a 100 point scale.


\subsubsection{Pair Project}
\label{\detokenize{content/0_course/assessments:pair-project}}
\sphinxAtStartPar
The second project can (but doesn’t have to) build on this first project. It is a team project that you will complete with a partner. It is meant to mimic the common practice of poster preparation and presentation. In a nutshell, you will conduct an original modeling project where you analytically and computationally model some E\&M phenomenon of your choosing, prepare a poster of the project, and present it to your classmates and me. In working on this project, you will be trying to answer the following questions:
\begin{itemize}
\item {} 
\sphinxAtStartPar
What is the area of E\&M that you are doing research on?

\item {} 
\sphinxAtStartPar
What are the questions that you are trying to answer about this area?

\item {} 
\sphinxAtStartPar
What theoretical models can be used to answer those questions?

\item {} 
\sphinxAtStartPar
What analytical and computational work did you do to answer those questions?

\item {} 
\sphinxAtStartPar
What were the resulting predictions that your work produced?

\item {} 
\sphinxAtStartPar
What are the limitations of what you have done? What are some remaining open questions?

\item {} 
\sphinxAtStartPar
What did each member of your pair contribute?

\end{itemize}

\sphinxAtStartPar
There will be six homework questions (10 if you count the 4 self\sphinxhyphen{}reflection homework questions) to help your team develop your poster project, Here is a \sphinxstyleemphasis{preliminary} listing of the homework questions that will appear (note that these are subject to change!):
\begin{itemize}
\item {} 
\sphinxAtStartPar
\sphinxstylestrong{Homework 8} \sphinxhyphen{} What are you and your partner proposing to do? What area of E\&M will you be conducting original calculations for? What source material are you drawing from? What has been done so far and what are you going to do? It’s ok if it’s a solved problem, but you will need to reproduce what has been done and extend it beyond what your reference material offers.

\item {} 
\sphinxAtStartPar
\sphinxstylestrong{Homework 9} \sphinxhyphen{} What is the plan for the next 5 weeks? How do you intend to structure the work? Explain the details of what will be done and who will be doing what. I expect 2\sphinxhyphen{}3 paragraphs describing the work and a detailed timeline.

\item {} 
\sphinxAtStartPar
\sphinxstylestrong{Homework 10} \sphinxhyphen{} Provide a detailed explanation of the models and theoretical calculations needed to set up your work. This should be presented as a “graphic” that would appear in a poster under “background or model.” There will also be a self\sphinxhyphen{}reflection homework problem \sphinxhyphen{} Who did what? What questions do you need to answer to continue to move forward? What help do you need from me or others?

\item {} 
\sphinxAtStartPar
\sphinxstylestrong{Homework 11} \sphinxhyphen{} There should be some sample calculations and figures produced by your code. This can be a notebook (Mathematica; Jupyter; etc.), but the work needs to be explained inline (i.e., what are you doing and why?). This is the work that is the meat of your original contribution. It need not be complete yet. There will also be a self\sphinxhyphen{}reflection homework problem \sphinxhyphen{} Who did what? What questions were you able to answer last week? What questions do you need to answer to continue to move forward? What help do you need from me or others?

\item {} 
\sphinxAtStartPar
\sphinxstylestrong{Homework 12} \sphinxhyphen{} At this point, you should produce draft figures for poster with captions. You should have pressed onward with your calculations and produced appropriate figures for your poster with captions. The “graphic” and caption should be turned in. These need not be complete in the sense that you should continue working on your calculations and models until the poster is turned in. There will also be a self\sphinxhyphen{}reflection homework problem \sphinxhyphen{} Who did what? What questions were you able to answer last week? What questions do you need to answer to continue to move forward? What help do you need from me or others?

\item {} 
\sphinxAtStartPar
\sphinxstylestrong{Homework 13} \sphinxhyphen{} Finally, you should produce an abstract for your poster. You should have a self\sphinxhyphen{}contained paragraph on what you will present in your poster. There will also be a self\sphinxhyphen{}reflection homework problem \sphinxhyphen{} Who did what? What questions were you able to answer last week? What questions do you need to answer to continue to move forward? What help do you need from me or others?

\end{itemize}

\sphinxAtStartPar
After Homework 13 (Tues. Apr 28 \sphinxhyphen{} during our final exam period), your poster is due. You and your teammate will present your poster to your classmates and myself. Your poster and presentation will be graded by me. But, you will also be given evaluation sheets for your classmates’ posters, which I’ll ask that you share with them (anonymously if you like). Your participation in the evaluation of your classmates’ posters counts towards your grade on your poster. There will also be a self\sphinxhyphen{}reflection/evaluation component to this assignment that asks you: Who did what? What did you learn? What did you want to learn more about? What was straight\sphinxhyphen{}forward? What was more difficult? Completion of this self\sphinxhyphen{}reflection/evaluation will also count towards the overall grade on your poster.


\subsubsection{Grading the pair project}
\label{\detokenize{content/0_course/assessments:grading-the-pair-project}}
\sphinxAtStartPar
A rubric for the individual project \DUrole{xref,myst}{appears here}. Notice that the rubric emphasizes several aspects of the poster with different weights. Each category will be scored on the following scale (4.0, 3.5, 3.0, 2.0, 0.0) and then averaged together using the weights for each category. Your final grade will be this averaged score converted to a 100 point scale.


\section{Classroom Environment}
\label{\detokenize{content/0_course/environment:classroom-environment}}\label{\detokenize{content/0_course/environment::doc}}

\subsection{Commitment to an Inclusive Classroom}
\label{\detokenize{content/0_course/environment:commitment-to-an-inclusive-classroom}}
\sphinxAtStartPar
I am committed to creating an inclusive classroom \sphinxhyphen{} one where you and your classmates
feel comfortable, intellectually challenged, and able to speak up about your ideas
and experiences. This means that our classroom, our virtual environments, and our interactions
need to be as inclusive as possible. Mutual respect, civility, and the ability to listen
and observe others are central to creating a classroom that is inclusive. I will strive to
do this and I ask that you do the same. If I can do anything to make the classroom a better
learning environment for you, please let me know.

\sphinxAtStartPar
\sphinxstylestrong{If you observe or experience behaviors that violate our commitment to inclusivity,
please let me know as soon as possible.}

\sphinxAtStartPar
If I violate this principle, please let me know
or please tell the undergraduate department chair,
Stuart Tessmer (\sphinxhref{mailto:tessmer@pa.msu.edu}{tessmer@pa.msu.edu}), who I have informed to tell me about any such incidents
without conveying student information to me.


\subsection{Comments on preparation:}
\label{\detokenize{content/0_course/environment:comments-on-preparation}}
\sphinxAtStartPar
Physics 482 covers material you might have seen before (Many of the topics
stem from PHY 184/294H material) but at a higher level of conceptual and
mathematical sophistication.

\sphinxAtStartPar
Therefore you should expect:
\begin{itemize}
\item {} 
\sphinxAtStartPar
a large amount of material covered quickly.

\item {} 
\sphinxAtStartPar
no recitations, and few examples covered in lecture. Most homework
problems are not similar to examples from class.

\item {} 
\sphinxAtStartPar
long, hard homework problems that usually cannot be completed by one
individual alone.

\item {} 
\sphinxAtStartPar
challenging projects.

\end{itemize}

\sphinxAtStartPar
Physics 482 is a challenging, upper‐division physics course. Unlike more
introductory courses, you are fully responsible for your own learning.
In particular, you control the pace of the course by asking questions in
class. I tend to speak quickly, and questions are important to slow down
the lecture. This means that if you don’t understand something, it is
your responsibility to ask questions. Attending class and the homework
help sessions gives you an opportunity to ask questions. I am here to
help you as much as possible, but I need your questions to know what you
don’t understand.

\sphinxAtStartPar
Physics 482 covers some of the most important physics and mathematical
methods in the field. Your reward for the hard work and effort will be
learning important and elegant material that you will use over and over
as a physics major. Here is what I have experienced, and heard from
other faculty teaching upper division physics in the past:
\begin{itemize}
\item {} 
\sphinxAtStartPar
most students reported spending a minimum of 10 hours per week on the
homework (!!)

\item {} 
\sphinxAtStartPar
students who didn’t attend the homework help sessions
often did poorly in the class.

\item {} 
\sphinxAtStartPar
students reported learning a tremendous amount in this class.

\end{itemize}

\sphinxAtStartPar
\sphinxstylestrong{The course topics that we will cover in Physics 482 are among the
greatest intellectual achievements of humans. Don’t be surprised if you
have to think hard and work hard to master the material.}


\section{Resources}
\label{\detokenize{content/0_course/resources:resources}}\label{\detokenize{content/0_course/resources::doc}}

\subsection{Confidentiality and Mandatory Reporting}
\label{\detokenize{content/0_course/resources:confidentiality-and-mandatory-reporting}}
\sphinxAtStartPar
College students often experience issues that may interfere with academic success such as academic stress, sleep problems, juggling responsibiities, life events, relationship concerns, or feelings of anxiety, hopelessness, or depression.
As your instructor, one of my responsibilities is to help create a safe learning environment and to support you through these situations and experiences.
I also have a mandatory reporting responsibility related to my role as a University employee.
It is my goal that you feel able to share information related to your life experiences in classroom
discussions, in written work, and in one\sphinxhyphen{}on\sphinxhyphen{}one meetings.
I will seek to keep information you share private to the greatest extent possible.
However, under Title IX, I am required to share information regarding sexual misconduct, relationship violence, or information
about criminal activity on MSU’s campus with the University including the Office of Institutional Equity (OIE).

\sphinxAtStartPar
\sphinxstylestrong{Students may speak to someone confidentially by contacting MSU Counseling and Psychiatric Service (CAPS) (\sphinxhref{http://caps.msu.edu}{caps.msu.edu}, 517\sphinxhyphen{}355\sphinxhyphen{}8270), MSU’s 24\sphinxhyphen{}hour Sexual Assault Crisis Line (\sphinxhref{http://endrape.msu.edu}{endrape.msu.edu}, 517\sphinxhyphen{}372\sphinxhyphen{}6666), or Olin Health Center (\sphinxhref{http://olin.msu.edu}{olin.msu.edu}, 517\sphinxhyphen{}884\sphinxhyphen{}6546).}


\subsection{Spartan Code of Honor Academic Pledge}
\label{\detokenize{content/0_course/resources:spartan-code-of-honor-academic-pledge}}
\sphinxAtStartPar
As a Spartan, I will strive to uphold values of the highest ethical standard. I will practice honesty in my work, foster honesty in my peers, and take pride in knowing that honor is worth more than grades. I will carry these values beyond my time as a student at Michigan State University, continuing the endeavor to build personal integrity in all that I do.


\subsection{Handling Emergency Situations}
\label{\detokenize{content/0_course/resources:handling-emergency-situations}}
\sphinxAtStartPar
\sphinxstyleemphasis{In the event of an emergency arising within the classroom, Prof. Caballero will notify you of what actions that may be required to ensure your safety. It is the responsibility of each student to understand the evacuation, “shelter\sphinxhyphen{}in\sphinxhyphen{}place,” and “secure\sphinxhyphen{}in\sphinxhyphen{}place” guidelines posted in each facility and to act in a safe manner. You are allowed to maintain cellular devices in a silent mode during this course, in order to receive emergency SMS text, phone or email messages distributed by the university. When anyone receives such a notification or observes an emergency situation, they should immediately bring it to the attention of Prof. Caballero in a way that causes the least disruption. If an evacuation is ordered, please ensure that you do it in a safe manner and facilitate those around you that may not otherwise be able to safely leave. When these orders are given, you do have the right as a member of this community to follow that order. Also, if a shelter\sphinxhyphen{}in\sphinxhyphen{}place or secure\sphinxhyphen{}in\sphinxhyphen{}place is ordered, please seek areas of refuge that are safe depending on the emergency encountered and provide assistance if it is advisable to do so.}


\chapter{What is Mathematical Modeling?}
\label{\detokenize{content/1_modeling/what_is_modeling:what-is-mathematical-modeling}}\label{\detokenize{content/1_modeling/what_is_modeling::doc}}
\sphinxAtStartPar
Nature reveals itself to us through interactions. We can tell from observations that it is nature’s interactions that lead to its evolution. How nature is changing and predicting how it will change in the future is the work of science. In this work, we observe nature and its interactions to make models of those observations. We aim to predict and explain our observations of nature through this building of models.

\sphinxAtStartPar
In physics, our goals are typically to explain and predict observations of physical phenomenon. Here, we focus ourselves to those canonical things that physicists concern themselves with: motion, fields, waves, atoms, nuclei, and so on.

\sphinxAtStartPar



\section{The Simple Harmonic Oscillator}
\label{\detokenize{content/1_modeling/SHO-intro:the-simple-harmonic-oscillator}}\label{\detokenize{content/1_modeling/SHO-intro::doc}}

\subsection{The SHO model}
\label{\detokenize{content/1_modeling/SHO-intro:the-sho-model}}
\sphinxAtStartPar
From these observations we find that a reasonable force model for a 1D spring system is given by:
\begin{equation*}
\begin{split}F_{spring} = -k |x-x_0|\end{split}
\end{equation*}
\sphinxAtStartPar
where \(k\) is the spring constant for the spring and \(x_0\) is the equilibrium location. As we know from Newton’s second law, if we can argue that this force is the only force or the dominant behavior we are trying to model in a mechanical situation, we have the following ordinary differential equation:
\begin{equation*}
\begin{split}F_{net} = F_{spring}\end{split}
\end{equation*}\begin{equation*}
\begin{split}m\ddot{x} = -k|x-x_0|\end{split}
\end{equation*}
\sphinxAtStartPar
Without much loss, we can recast the problem in terms of the distance from equilbirium (\(s=|x-x_0|\)). \sphinxstyleemphasis{As you will learn this is not a problem from vertically hanging springs near the surface of the Earth because the graviational force is constant and depends on the mass of the hanging weight.}

\begin{sphinxadmonition}{note}{The SHO mathematical model}
\begin{equation*}
\begin{split}m\ddot{s} = -ks\end{split}
\end{equation*}\end{sphinxadmonition}


\subsubsection{Finding the general solution}
\label{\detokenize{content/1_modeling/SHO-intro:finding-the-general-solution}}
\sphinxAtStartPar
We can solve this using our typical approach to 2D linear ODEs; guess the form of the solution and check it.
\begin{equation*}
\begin{split}\ddot{s} = -\frac{k}{m}s\end{split}
\end{equation*}
\sphinxAtStartPar
Assume: \(s(t) = A\exp(i\omega t)\) As we will learn this form is a good guess for linear ODEs. This gives:
\begin{equation*}
\begin{split}\ddot{s} = -\omega^2\;A\exp(i\omega t)\end{split}
\end{equation*}
\sphinxAtStartPar
So that,
\begin{equation*}
\begin{split}-\omega^2\;A\exp(i\omega t) = -\frac{k}{m}A\exp(i\omega t)\end{split}
\end{equation*}
\sphinxAtStartPar
This solution is consistent with the model (i.e., solves the differential equation) if we set \(\omega\) and notice it is exclusively positive:
\begin{equation*}
\begin{split}\omega^2 = \frac{k}{m} > 0\end{split}
\end{equation*}
\sphinxAtStartPar
Thus, \(s(t) = A\exp(i\sqrt{\frac{k}{m}} t)\) is \sphinxstylestrong{a solution} to this problem. Notice we have to set \(A\) using initial conditions for the problem. This is called a \sphinxstylestrong{initial value problem} in mathematics. You likely have heard about the Euler formulation; it is one of the most important relationships in mathematics {[}\hyperlink{cite.content/1_modeling/SHO-intro:id2}{Nahin, 2011}, \hyperlink{cite.content/1_modeling/SHO-intro:id3}{Stipp, 2017}{]}. We will get into it more later. It appears below:

\begin{sphinxadmonition}{note}{Leonhard Euler’s famous formula}
\begin{equation*}
\begin{split}\exp{i\theta} = \cos{\theta} + i \sin{\theta}\end{split}
\end{equation*}\end{sphinxadmonition}

\sphinxAtStartPar
Thus, we know that our solution gives rise to real sinusoidal solutions of the form:
\begin{equation*}
\begin{split}s(t) = A\left(cos{\sqrt{\frac{k}{m}} t} + i\sin{\sqrt{\frac{k}{m}} t}\right)\end{split}
\end{equation*}
\sphinxAtStartPar
This probably doesn’t look quite right…there’s an \(i\) in the solution for real physical motion! We will get to that, but we need the initial conditions to do so. More importantly, there’s only one free parameter in our current solution, \(A\). Given that this equation is \sphinxstylestrong{a solution, is it the only solution?}


\subsubsection{The order of your differential equation matters}
\label{\detokenize{content/1_modeling/SHO-intro:the-order-of-your-differential-equation-matters}}
\sphinxAtStartPar
You probably already know that our general solution for the SHO is:
\begin{equation*}
\begin{split}s(t) = A\exp(i\sqrt{\frac{k}{m}} t) + B\exp(-i\sqrt{\frac{k}{m}} t)\end{split}
\end{equation*}
\sphinxAtStartPar
Or written in a way that is more often presented:
\begin{equation*}
\begin{split}s(t) = A'\cos(\sqrt{\frac{k}{m}} t) + B'\sin(\sqrt{\frac{k}{m}} t)\end{split}
\end{equation*}
\sphinxAtStartPar
where we have made clear that the free parameters \(A\) and \(B\) are not equal to \(A'\) and \(B'\), respectively. Regardless, notice there’s two free parameters in these formula. That is because we are solving a \sphinxstylestrong{two dimensional} linear ODE. Because there’s two derivatives on \(s\) (i.e., \(\ddot{s}\)), we expect to have two free parameters that are set by the intial conditions. This is characteristic of an \sphinxstylestrong{initial value problem}.

\begin{sphinxadmonition}{note}{Important Tip}

\sphinxAtStartPar
The number of free parameters that are set by problem conditions in an initial value problem is determined by the order of the differential equation.

\sphinxAtStartPar
Second order? Two conditions needed. Two free parameters to determine. Fourth order? Four conditions needed. Four free parameters to determine.
\end{sphinxadmonition}


\section{Jupyter Notebook files}
\label{\detokenize{content/1_modeling/notebooks:jupyter-notebook-files}}\label{\detokenize{content/1_modeling/notebooks:file-types-notebooks}}\label{\detokenize{content/1_modeling/notebooks::doc}}
\sphinxAtStartPar
You can create content with Jupyter notebooks.
For example, the content for the current page is contained in \sphinxcode{\sphinxupquote{this notebook file}}.

\begin{sphinxShadowBox}
\sphinxstylesidebartitle{}

\sphinxAtStartPar
If you’d like to write in plain\sphinxhyphen{}text files, but still keep a notebook structure, you can write
Jupyter notebooks with MyST Markdown, which are then automatically converted to notebooks.
See \DUrole{xref,myst}{} for more details.
\end{sphinxShadowBox}

\sphinxAtStartPar
Jupyter Book supports all Markdown that is supported by Jupyter Notebook.
This is mostly a flavour of Markdown called \sphinxhref{https://commonmark.org/}{CommonMark Markdown} with minor modifications.
For more information about writing Jupyter\sphinxhyphen{}flavoured Markdown in Jupyter Book, see \DUrole{xref,myst}{}.


\subsection{Code blocks and image outputs}
\label{\detokenize{content/1_modeling/notebooks:code-blocks-and-image-outputs}}
\sphinxAtStartPar
Jupyter Book will also embed your code blocks and output in your book.
For example, here’s some sample Matplotlib code:

\begin{sphinxuseclass}{cell}\begin{sphinxVerbatimInput}

\begin{sphinxuseclass}{cell_input}
\begin{sphinxVerbatim}[commandchars=\\\{\}]
\PYG{k+kn}{from} \PYG{n+nn}{matplotlib} \PYG{k+kn}{import} \PYG{n}{rcParams}\PYG{p}{,} \PYG{n}{cycler}
\PYG{k+kn}{import} \PYG{n+nn}{matplotlib}\PYG{n+nn}{.}\PYG{n+nn}{pyplot} \PYG{k}{as} \PYG{n+nn}{plt}
\PYG{k+kn}{import} \PYG{n+nn}{numpy} \PYG{k}{as} \PYG{n+nn}{np}
\PYG{n}{plt}\PYG{o}{.}\PYG{n}{ion}\PYG{p}{(}\PYG{p}{)}
\end{sphinxVerbatim}

\end{sphinxuseclass}\end{sphinxVerbatimInput}
\begin{sphinxVerbatimOutput}

\begin{sphinxuseclass}{cell_output}
\begin{sphinxVerbatim}[commandchars=\\\{\}]
\PYG{g+gt}{\PYGZhy{}\PYGZhy{}\PYGZhy{}\PYGZhy{}\PYGZhy{}\PYGZhy{}\PYGZhy{}\PYGZhy{}\PYGZhy{}\PYGZhy{}\PYGZhy{}\PYGZhy{}\PYGZhy{}\PYGZhy{}\PYGZhy{}\PYGZhy{}\PYGZhy{}\PYGZhy{}\PYGZhy{}\PYGZhy{}\PYGZhy{}\PYGZhy{}\PYGZhy{}\PYGZhy{}\PYGZhy{}\PYGZhy{}\PYGZhy{}\PYGZhy{}\PYGZhy{}\PYGZhy{}\PYGZhy{}\PYGZhy{}\PYGZhy{}\PYGZhy{}\PYGZhy{}\PYGZhy{}\PYGZhy{}\PYGZhy{}\PYGZhy{}\PYGZhy{}\PYGZhy{}\PYGZhy{}\PYGZhy{}\PYGZhy{}\PYGZhy{}\PYGZhy{}\PYGZhy{}\PYGZhy{}\PYGZhy{}\PYGZhy{}\PYGZhy{}\PYGZhy{}\PYGZhy{}\PYGZhy{}\PYGZhy{}\PYGZhy{}\PYGZhy{}\PYGZhy{}\PYGZhy{}\PYGZhy{}\PYGZhy{}\PYGZhy{}\PYGZhy{}\PYGZhy{}\PYGZhy{}\PYGZhy{}\PYGZhy{}\PYGZhy{}\PYGZhy{}\PYGZhy{}\PYGZhy{}\PYGZhy{}\PYGZhy{}\PYGZhy{}\PYGZhy{}}
\PYG{n+ne}{ModuleNotFoundError}\PYG{g+gWhitespace}{                       }Traceback (most recent call last)
\PYG{n}{Input} \PYG{n}{In} \PYG{p}{[}\PYG{l+m+mi}{1}\PYG{p}{]}\PYG{p}{,} \PYG{o+ow}{in} \PYG{o}{\PYGZlt{}}\PYG{n}{module}\PYG{o}{\PYGZgt{}}
\PYG{n+ne}{\PYGZhy{}\PYGZhy{}\PYGZhy{}\PYGZhy{}\PYGZgt{} }\PYG{l+m+mi}{1} \PYG{k+kn}{from} \PYG{n+nn}{matplotlib} \PYG{k+kn}{import} \PYG{n}{rcParams}\PYG{p}{,} \PYG{n}{cycler}
\PYG{g+gWhitespace}{      }\PYG{l+m+mi}{2} \PYG{k+kn}{import} \PYG{n+nn}{matplotlib}\PYG{n+nn}{.}\PYG{n+nn}{pyplot} \PYG{k}{as} \PYG{n+nn}{plt}
\PYG{g+gWhitespace}{      }\PYG{l+m+mi}{3} \PYG{k+kn}{import} \PYG{n+nn}{numpy} \PYG{k}{as} \PYG{n+nn}{np}

\PYG{n+ne}{ModuleNotFoundError}: No module named \PYGZsq{}matplotlib\PYGZsq{}
\end{sphinxVerbatim}

\end{sphinxuseclass}\end{sphinxVerbatimOutput}

\end{sphinxuseclass}
\begin{sphinxuseclass}{cell}
\begin{sphinxuseclass}{tag_remove-stdout}\begin{sphinxVerbatimInput}

\begin{sphinxuseclass}{cell_input}
\begin{sphinxVerbatim}[commandchars=\\\{\}]
\PYG{c+c1}{\PYGZsh{} Fixing random state for reproducibility}
\PYG{n}{np}\PYG{o}{.}\PYG{n}{random}\PYG{o}{.}\PYG{n}{seed}\PYG{p}{(}\PYG{l+m+mi}{19680801}\PYG{p}{)}

\PYG{n}{N} \PYG{o}{=} \PYG{l+m+mi}{10}
\PYG{n}{data} \PYG{o}{=} \PYG{p}{[}\PYG{n}{np}\PYG{o}{.}\PYG{n}{logspace}\PYG{p}{(}\PYG{l+m+mi}{0}\PYG{p}{,} \PYG{l+m+mi}{1}\PYG{p}{,} \PYG{l+m+mi}{100}\PYG{p}{)} \PYG{o}{+} \PYG{n}{np}\PYG{o}{.}\PYG{n}{random}\PYG{o}{.}\PYG{n}{randn}\PYG{p}{(}\PYG{l+m+mi}{100}\PYG{p}{)} \PYG{o}{+} \PYG{n}{ii} \PYG{k}{for} \PYG{n}{ii} \PYG{o+ow}{in} \PYG{n+nb}{range}\PYG{p}{(}\PYG{n}{N}\PYG{p}{)}\PYG{p}{]}
\PYG{n}{data} \PYG{o}{=} \PYG{n}{np}\PYG{o}{.}\PYG{n}{array}\PYG{p}{(}\PYG{n}{data}\PYG{p}{)}\PYG{o}{.}\PYG{n}{T}
\PYG{n}{cmap} \PYG{o}{=} \PYG{n}{plt}\PYG{o}{.}\PYG{n}{cm}\PYG{o}{.}\PYG{n}{coolwarm}
\PYG{n}{rcParams}\PYG{p}{[}\PYG{l+s+s1}{\PYGZsq{}}\PYG{l+s+s1}{axes.prop\PYGZus{}cycle}\PYG{l+s+s1}{\PYGZsq{}}\PYG{p}{]} \PYG{o}{=} \PYG{n}{cycler}\PYG{p}{(}\PYG{n}{color}\PYG{o}{=}\PYG{n}{cmap}\PYG{p}{(}\PYG{n}{np}\PYG{o}{.}\PYG{n}{linspace}\PYG{p}{(}\PYG{l+m+mi}{0}\PYG{p}{,} \PYG{l+m+mi}{1}\PYG{p}{,} \PYG{n}{N}\PYG{p}{)}\PYG{p}{)}\PYG{p}{)}


\PYG{k+kn}{from} \PYG{n+nn}{matplotlib}\PYG{n+nn}{.}\PYG{n+nn}{lines} \PYG{k+kn}{import} \PYG{n}{Line2D}
\PYG{n}{custom\PYGZus{}lines} \PYG{o}{=} \PYG{p}{[}\PYG{n}{Line2D}\PYG{p}{(}\PYG{p}{[}\PYG{l+m+mi}{0}\PYG{p}{]}\PYG{p}{,} \PYG{p}{[}\PYG{l+m+mi}{0}\PYG{p}{]}\PYG{p}{,} \PYG{n}{color}\PYG{o}{=}\PYG{n}{cmap}\PYG{p}{(}\PYG{l+m+mf}{0.}\PYG{p}{)}\PYG{p}{,} \PYG{n}{lw}\PYG{o}{=}\PYG{l+m+mi}{4}\PYG{p}{)}\PYG{p}{,}
                \PYG{n}{Line2D}\PYG{p}{(}\PYG{p}{[}\PYG{l+m+mi}{0}\PYG{p}{]}\PYG{p}{,} \PYG{p}{[}\PYG{l+m+mi}{0}\PYG{p}{]}\PYG{p}{,} \PYG{n}{color}\PYG{o}{=}\PYG{n}{cmap}\PYG{p}{(}\PYG{l+m+mf}{.5}\PYG{p}{)}\PYG{p}{,} \PYG{n}{lw}\PYG{o}{=}\PYG{l+m+mi}{4}\PYG{p}{)}\PYG{p}{,}
                \PYG{n}{Line2D}\PYG{p}{(}\PYG{p}{[}\PYG{l+m+mi}{0}\PYG{p}{]}\PYG{p}{,} \PYG{p}{[}\PYG{l+m+mi}{0}\PYG{p}{]}\PYG{p}{,} \PYG{n}{color}\PYG{o}{=}\PYG{n}{cmap}\PYG{p}{(}\PYG{l+m+mf}{1.}\PYG{p}{)}\PYG{p}{,} \PYG{n}{lw}\PYG{o}{=}\PYG{l+m+mi}{4}\PYG{p}{)}\PYG{p}{]}

\PYG{n}{fig}\PYG{p}{,} \PYG{n}{ax} \PYG{o}{=} \PYG{n}{plt}\PYG{o}{.}\PYG{n}{subplots}\PYG{p}{(}\PYG{n}{figsize}\PYG{o}{=}\PYG{p}{(}\PYG{l+m+mi}{10}\PYG{p}{,} \PYG{l+m+mi}{5}\PYG{p}{)}\PYG{p}{)}
\PYG{n}{lines} \PYG{o}{=} \PYG{n}{ax}\PYG{o}{.}\PYG{n}{plot}\PYG{p}{(}\PYG{n}{data}\PYG{p}{)}
\PYG{n}{ax}\PYG{o}{.}\PYG{n}{legend}\PYG{p}{(}\PYG{n}{custom\PYGZus{}lines}\PYG{p}{,} \PYG{p}{[}\PYG{l+s+s1}{\PYGZsq{}}\PYG{l+s+s1}{Cold}\PYG{l+s+s1}{\PYGZsq{}}\PYG{p}{,} \PYG{l+s+s1}{\PYGZsq{}}\PYG{l+s+s1}{Medium}\PYG{l+s+s1}{\PYGZsq{}}\PYG{p}{,} \PYG{l+s+s1}{\PYGZsq{}}\PYG{l+s+s1}{Hot}\PYG{l+s+s1}{\PYGZsq{}}\PYG{p}{]}\PYG{p}{)}\PYG{p}{;}
\end{sphinxVerbatim}

\end{sphinxuseclass}\end{sphinxVerbatimInput}

\end{sphinxuseclass}
\end{sphinxuseclass}
\sphinxAtStartPar
Note that the image above is captured and displayed in your site.

\begin{sphinxuseclass}{cell}
\begin{sphinxuseclass}{tag_popout}
\begin{sphinxuseclass}{tag_remove-input}
\begin{sphinxuseclass}{tag_remove-stdout}
\end{sphinxuseclass}
\end{sphinxuseclass}
\end{sphinxuseclass}
\end{sphinxuseclass}
\begin{sphinxShadowBox}
\sphinxstylesidebartitle{\sphinxstylestrong{You can also pop out content to the side!}}

\sphinxAtStartPar
For more information on how to do this,
check out the \DUrole{xref,std,std-ref}{layout/sidebar} section.
\end{sphinxShadowBox}


\subsection{Removing content before publishing}
\label{\detokenize{content/1_modeling/notebooks:removing-content-before-publishing}}
\sphinxAtStartPar
You can also remove some content before publishing your book to the web.
For reference, \sphinxcode{\sphinxupquote{you can download the notebook content for this page}}.

\sphinxAtStartPar
You can \sphinxstylestrong{remove only the code} so that images and other output still show up.

\begin{sphinxuseclass}{cell}
\begin{sphinxuseclass}{tag_hide-input}
\end{sphinxuseclass}
\end{sphinxuseclass}
\sphinxAtStartPar
Which works well if you’d like to quickly display cell output without cluttering your content with code.
This works for any cell output, like a Pandas DataFrame.

\begin{sphinxuseclass}{cell}
\begin{sphinxuseclass}{tag_hide-input}
\end{sphinxuseclass}
\end{sphinxuseclass}
\sphinxAtStartPar
See \DUrole{xref,std,std-ref}{hiding/remove\sphinxhyphen{}content} for more information about hiding and removing content.


\subsection{Interactive outputs}
\label{\detokenize{content/1_modeling/notebooks:interactive-outputs}}
\sphinxAtStartPar
We can do the same for \sphinxstyleemphasis{interactive} material. Below we’ll display a map
using \sphinxhref{https://python-visualization.github.io/folium/}{folium}. When your book is built,
the code for creating the interactive map is retained.

\begin{sphinxShadowBox}
\sphinxstylesidebartitle{}

\sphinxAtStartPar
\sphinxstylestrong{This will only work for some packages.} They need to be able to output standalone
HTML/Javascript, and not
depend on an underlying Python kernel to work.
\end{sphinxShadowBox}

\begin{sphinxuseclass}{cell}\begin{sphinxVerbatimInput}

\begin{sphinxuseclass}{cell_input}
\begin{sphinxVerbatim}[commandchars=\\\{\}]
\PYG{k+kn}{import} \PYG{n+nn}{folium}
\PYG{n}{m} \PYG{o}{=} \PYG{n}{folium}\PYG{o}{.}\PYG{n}{Map}\PYG{p}{(}
    \PYG{n}{location}\PYG{o}{=}\PYG{p}{[}\PYG{l+m+mf}{45.372}\PYG{p}{,} \PYG{o}{\PYGZhy{}}\PYG{l+m+mf}{121.6972}\PYG{p}{]}\PYG{p}{,}
    \PYG{n}{zoom\PYGZus{}start}\PYG{o}{=}\PYG{l+m+mi}{12}\PYG{p}{,}
    \PYG{n}{tiles}\PYG{o}{=}\PYG{l+s+s1}{\PYGZsq{}}\PYG{l+s+s1}{Stamen Terrain}\PYG{l+s+s1}{\PYGZsq{}}
\PYG{p}{)}

\PYG{n}{folium}\PYG{o}{.}\PYG{n}{Marker}\PYG{p}{(}
    \PYG{n}{location}\PYG{o}{=}\PYG{p}{[}\PYG{l+m+mf}{45.3288}\PYG{p}{,} \PYG{o}{\PYGZhy{}}\PYG{l+m+mf}{121.6625}\PYG{p}{]}\PYG{p}{,}
    \PYG{n}{popup}\PYG{o}{=}\PYG{l+s+s1}{\PYGZsq{}}\PYG{l+s+s1}{Mt. Hood Meadows}\PYG{l+s+s1}{\PYGZsq{}}\PYG{p}{,}
    \PYG{n}{icon}\PYG{o}{=}\PYG{n}{folium}\PYG{o}{.}\PYG{n}{Icon}\PYG{p}{(}\PYG{n}{icon}\PYG{o}{=}\PYG{l+s+s1}{\PYGZsq{}}\PYG{l+s+s1}{cloud}\PYG{l+s+s1}{\PYGZsq{}}\PYG{p}{)}
\PYG{p}{)}\PYG{o}{.}\PYG{n}{add\PYGZus{}to}\PYG{p}{(}\PYG{n}{m}\PYG{p}{)}

\PYG{n}{folium}\PYG{o}{.}\PYG{n}{Marker}\PYG{p}{(}
    \PYG{n}{location}\PYG{o}{=}\PYG{p}{[}\PYG{l+m+mf}{45.3311}\PYG{p}{,} \PYG{o}{\PYGZhy{}}\PYG{l+m+mf}{121.7113}\PYG{p}{]}\PYG{p}{,}
    \PYG{n}{popup}\PYG{o}{=}\PYG{l+s+s1}{\PYGZsq{}}\PYG{l+s+s1}{Timberline Lodge}\PYG{l+s+s1}{\PYGZsq{}}\PYG{p}{,}
    \PYG{n}{icon}\PYG{o}{=}\PYG{n}{folium}\PYG{o}{.}\PYG{n}{Icon}\PYG{p}{(}\PYG{n}{color}\PYG{o}{=}\PYG{l+s+s1}{\PYGZsq{}}\PYG{l+s+s1}{green}\PYG{l+s+s1}{\PYGZsq{}}\PYG{p}{)}
\PYG{p}{)}\PYG{o}{.}\PYG{n}{add\PYGZus{}to}\PYG{p}{(}\PYG{n}{m}\PYG{p}{)}

\PYG{n}{folium}\PYG{o}{.}\PYG{n}{Marker}\PYG{p}{(}
    \PYG{n}{location}\PYG{o}{=}\PYG{p}{[}\PYG{l+m+mf}{45.3300}\PYG{p}{,} \PYG{o}{\PYGZhy{}}\PYG{l+m+mf}{121.6823}\PYG{p}{]}\PYG{p}{,}
    \PYG{n}{popup}\PYG{o}{=}\PYG{l+s+s1}{\PYGZsq{}}\PYG{l+s+s1}{Some Other Location}\PYG{l+s+s1}{\PYGZsq{}}\PYG{p}{,}
    \PYG{n}{icon}\PYG{o}{=}\PYG{n}{folium}\PYG{o}{.}\PYG{n}{Icon}\PYG{p}{(}\PYG{n}{color}\PYG{o}{=}\PYG{l+s+s1}{\PYGZsq{}}\PYG{l+s+s1}{red}\PYG{l+s+s1}{\PYGZsq{}}\PYG{p}{,} \PYG{n}{icon}\PYG{o}{=}\PYG{l+s+s1}{\PYGZsq{}}\PYG{l+s+s1}{info\PYGZhy{}sign}\PYG{l+s+s1}{\PYGZsq{}}\PYG{p}{)}
\PYG{p}{)}\PYG{o}{.}\PYG{n}{add\PYGZus{}to}\PYG{p}{(}\PYG{n}{m}\PYG{p}{)}

\PYG{n}{m}
\end{sphinxVerbatim}

\end{sphinxuseclass}\end{sphinxVerbatimInput}

\end{sphinxuseclass}

\subsection{Rich outputs from notebook cells}
\label{\detokenize{content/1_modeling/notebooks:rich-outputs-from-notebook-cells}}
\sphinxAtStartPar
Because notebooks have rich text outputs, you can store these in
your Jupyter Book as well! For example, here is the command line help
menu, see how it is nicely formatted.

\begin{sphinxuseclass}{cell}\begin{sphinxVerbatimInput}

\begin{sphinxuseclass}{cell_input}
\begin{sphinxVerbatim}[commandchars=\\\{\}]
\PYG{o}{!}jupyter\PYGZhy{}book build \PYGZhy{}\PYGZhy{}help
\end{sphinxVerbatim}

\end{sphinxuseclass}\end{sphinxVerbatimInput}

\end{sphinxuseclass}
\sphinxAtStartPar
And here is an error. You can mark notebook cells as “expected to error” by adding a
\sphinxcode{\sphinxupquote{raises\sphinxhyphen{}exception}} tag to them.

\begin{sphinxuseclass}{cell}
\begin{sphinxuseclass}{tag_raises-exception}\begin{sphinxVerbatimInput}

\begin{sphinxuseclass}{cell_input}
\begin{sphinxVerbatim}[commandchars=\\\{\}]
\PYG{n}{this\PYGZus{}will\PYGZus{}error}
\end{sphinxVerbatim}

\end{sphinxuseclass}\end{sphinxVerbatimInput}

\end{sphinxuseclass}
\end{sphinxuseclass}

\subsection{More features with Jupyter notebooks}
\label{\detokenize{content/1_modeling/notebooks:more-features-with-jupyter-notebooks}}
\sphinxAtStartPar
There are many other features of Jupyter notebooks to take advantage of,
such as automatically generating Binder links for notebooks or connecting your content with a kernel in the cloud.
For more information browse the pages in this site, and \DUrole{xref,myst}{} in particular.

\begin{sphinxthebibliography}{1}
\bibitem[1]{content/1_modeling/SHO-intro:id2}
\sphinxAtStartPar
Paul J Nahin. \sphinxstyleemphasis{Dr. Euler's fabulous formula}. Princeton University Press, 2011.
\bibitem[2]{content/1_modeling/SHO-intro:id3}
\sphinxAtStartPar
David Stipp. \sphinxstyleemphasis{A most elegant equation: Euler's formula and the beauty of mathematics}. Hachette UK, 2017.
\end{sphinxthebibliography}







\renewcommand{\indexname}{Index}
\printindex
\end{document}