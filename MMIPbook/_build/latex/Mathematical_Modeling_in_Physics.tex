%% Generated by Sphinx.
\def\sphinxdocclass{jupyterBook}
\documentclass[letterpaper,10pt,english]{jupyterBook}
\ifdefined\pdfpxdimen
   \let\sphinxpxdimen\pdfpxdimen\else\newdimen\sphinxpxdimen
\fi \sphinxpxdimen=.75bp\relax
\ifdefined\pdfimageresolution
    \pdfimageresolution= \numexpr \dimexpr1in\relax/\sphinxpxdimen\relax
\fi
%% let collapsible pdf bookmarks panel have high depth per default
\PassOptionsToPackage{bookmarksdepth=5}{hyperref}
%% turn off hyperref patch of \index as sphinx.xdy xindy module takes care of
%% suitable \hyperpage mark-up, working around hyperref-xindy incompatibility
\PassOptionsToPackage{hyperindex=false}{hyperref}
%% memoir class requires extra handling
\makeatletter\@ifclassloaded{memoir}
{\ifdefined\memhyperindexfalse\memhyperindexfalse\fi}{}\makeatother

\PassOptionsToPackage{warn}{textcomp}

\catcode`^^^^00a0\active\protected\def^^^^00a0{\leavevmode\nobreak\ }
\usepackage{cmap}
\usepackage{fontspec}
\defaultfontfeatures[\rmfamily,\sffamily,\ttfamily]{}
\usepackage{amsmath,amssymb,amstext}
\usepackage{polyglossia}
\setmainlanguage{english}



\setmainfont{FreeSerif}[
  Extension      = .otf,
  UprightFont    = *,
  ItalicFont     = *Italic,
  BoldFont       = *Bold,
  BoldItalicFont = *BoldItalic
]
\setsansfont{FreeSans}[
  Extension      = .otf,
  UprightFont    = *,
  ItalicFont     = *Oblique,
  BoldFont       = *Bold,
  BoldItalicFont = *BoldOblique,
]
\setmonofont{FreeMono}[
  Extension      = .otf,
  UprightFont    = *,
  ItalicFont     = *Oblique,
  BoldFont       = *Bold,
  BoldItalicFont = *BoldOblique,
]



\usepackage[Bjarne]{fncychap}
\usepackage[,numfigreset=1,mathnumfig]{sphinx}

\fvset{fontsize=\small}
\usepackage{geometry}


% Include hyperref last.
\usepackage{hyperref}
% Fix anchor placement for figures with captions.
\usepackage{hypcap}% it must be loaded after hyperref.
% Set up styles of URL: it should be placed after hyperref.
\urlstyle{same}


\usepackage{sphinxmessages}



        % Start of preamble defined in sphinx-jupyterbook-latex %
         \usepackage[Latin,Greek]{ucharclasses}
        \usepackage{unicode-math}
        % fixing title of the toc
        \addto\captionsenglish{\renewcommand{\contentsname}{Contents}}
        \hypersetup{
            pdfencoding=auto,
            psdextra
        }
        % End of preamble defined in sphinx-jupyterbook-latex %
        

\title{Mathematical Modeling in Physics}
\date{Feb 24, 2022}
\release{}
\author{Danny Caballero}
\newcommand{\sphinxlogo}{\vbox{}}
\renewcommand{\releasename}{}
\makeindex
\begin{document}

\pagestyle{empty}
\sphinxmaketitle
\pagestyle{plain}
\sphinxtableofcontents
\pagestyle{normal}
\phantomsection\label{\detokenize{content/intro::doc}}
\sphinxAtStartPar
\sphinxincludegraphics{{tc_big}.jpg}



\sphinxAtStartPar
In this course, you will learn to:
\begin{itemize}
\item {} 
\sphinxAtStartPar
Develop models and stuff

\item {} 
\sphinxAtStartPar
Collaborate

\end{itemize}

\sphinxAtStartPar
The rest of this JupyterBook is (currently) organized as follows:
\begin{itemize}
\item {} 
\sphinxAtStartPar
{\hyperref[\detokenize{content/0_course/syllabus::doc}]{\sphinxcrossref{Syllabus}}}

\item {} 
\sphinxAtStartPar
{\hyperref[\detokenize{content/1_modeling/what_is_modeling::doc}]{\sphinxcrossref{What is Modeling?}}}

\end{itemize}


\chapter{Syllabus}
\label{\detokenize{content/0_course/syllabus:syllabus}}\label{\detokenize{content/0_course/syllabus::doc}}

\section{Course Goals}
\label{\detokenize{content/0_course/goals:course-goals}}\label{\detokenize{content/0_course/goals::doc}}

\section{Course Design}
\label{\detokenize{content/0_course/design:course-design}}\label{\detokenize{content/0_course/design::doc}}

\section{Assessments}
\label{\detokenize{content/0_course/assessments:assessments}}\label{\detokenize{content/0_course/assessments::doc}}

\section{Classroom Environment}
\label{\detokenize{content/0_course/environment:classroom-environment}}\label{\detokenize{content/0_course/environment::doc}}

\section{Resources}
\label{\detokenize{content/0_course/resources:resources}}\label{\detokenize{content/0_course/resources::doc}}

\chapter{What is Modeling?}
\label{\detokenize{content/1_modeling/what_is_modeling:what-is-modeling}}\label{\detokenize{content/1_modeling/what_is_modeling::doc}}






\renewcommand{\indexname}{Index}
\printindex
\end{document}